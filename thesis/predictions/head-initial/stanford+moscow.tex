\subsubsection{Podejścia stanfordzkie i~moskiewskie}

\begin{table}[h]
\centering
\begin{tabular}{lclc}

(L-L) & 
\begin{dependency}[hide label, edge unit distance=0.5ex, baseline=-\the\dimexpr\fontdimen22\textfont2\relax]
        \begin{deptext}
        $\odot$\&$\square$\&$\square$\&$\square$\&$\boxdot$\&$\square$\&$\square$\&$\square$\&$\square$\&$\square$\&$\square$\\
            \end{deptext}
	  \depedge{1}{2}{}
	  \depedge{2}{6}{}
	  \depedge{6}{5}{}
            \wordgroup{1}{2}{4}{L}
            \wordgroup{1}{6}{11}{R}
        \end{dependency}

& (L-R) &

\begin{dependency}[hide label, edge unit distance=0.5ex, baseline=-\the\dimexpr\fontdimen22\textfont2\relax]
        \begin{deptext}
        $\odot$\&$\square$\&$\square$\&$\square$\&$\square$\&$\square$\&$\square$\&$\boxdot$\&$\square$\&$\square$\&$\square$\\
            \end{deptext}
	  \depedge{1}{2}{}
	  \depedge{2}{9}{}
	  \depedge{9}{8}{}
            \wordgroup{1}{2}{7}{L}
            \wordgroup{1}{9}{11}{R}
        \end{dependency}
        
\\ (0-L) &

\begin{dependency}[hide label, edge unit distance=0.5ex, baseline=-\the\dimexpr\fontdimen22\textfont2\relax]
        \begin{deptext}
        $\square$\&$\square$\&$\square$\&$\boxdot$\&$\square$\&$\square$\&$\square$\&$\square$\&$\square$\&$\square$\\
            \end{deptext}
	  \depedge{5}{4}{}
	  \depedge{1}{5}{}
            \wordgroup{1}{1}{3}{L}
            \wordgroup{1}{5}{10}{R}
        \end{dependency}
        
& (0-R) &

\begin{dependency}[hide label, edge unit distance=0.5ex, baseline=-\the\dimexpr\fontdimen22\textfont2\relax]
        \begin{deptext}
        $\square$\&$\square$\&$\square$\&$\square$\&$\square$\&$\square$\&$\boxdot$\&$\square$\&$\square$\&$\square$\\
            \end{deptext}
	  \depedge{8}{7}{}
	  \depedge{1}{8}{}
            \wordgroup{1}{1}{6}{L}
            \wordgroup{1}{8}{10}{R}
        \end{dependency}

\\ (R-L) &

\begin{dependency}[hide label, edge unit distance=0.5ex, baseline=-\the\dimexpr\fontdimen22\textfont2\relax]
        \begin{deptext}
        $\square$\&$\square$\&$\square$\&$\boxdot$\&$\square$\&$\square$\&$\square$\&$\square$\&$\square$\&$\square$\&$\odot$\\
            \end{deptext}
	  \depedge{11}{1}{}
	  \depedge{1}{5}{}
	  \depedge{5}{4}{}
            \wordgroup{1}{1}{3}{L}
            \wordgroup{1}{5}{10}{R}
        \end{dependency}
        
& (R-R) &

\begin{dependency}[hide label, edge unit distance=0.5ex,  baseline=-\the\dimexpr\fontdimen22\textfont2\relax]
        \begin{deptext}
        $\square$\&$\square$\&$\square$\&$\square$\&$\square$\&$\square$\&$\boxdot$\&$\square$\&$\square$\&$\square$\&$\odot$\\
            \end{deptext}
	  \depedge{11}{1}{}
	  \depedge{1}{8}{}
	  \depedge{8}{7}{}
            \wordgroup{1}{1}{6}{L}
            \wordgroup{1}{8}{10}{R}
        \end{dependency}
        
\\
\end{tabular}
\end{table}

Model stanfordzki przewiduje, że wraz~ze wzrostem różnicy długości członów:
\begin{itemize}
\item odsetek koordynacji (L-L) względem wszystkich koordynacji (L) \textbf{rośnie};
\item odsetek koordynacji (0-L) względem wszystkich koordynacji (0) \textbf{rośnie};
\item odsetek koordynacji (R-L) względem wszystkich koordynacji (R) \textbf{rośnie}.
\end{itemize}

W~przypadku koordynacji dwuczłonowych podejście moskiewskie przewiduje bardzo podobną strukturę koordynacji, co podejście stanfordzkie. Modele te różnią się jedynie dwiema krawędziami -- jedną, łączącą głowę lewego członu z~głową prawego członu oraz drugą, łączącą głowę prawego członu ze~spójnikiem koordynacji. Różnice te nie mają istotnego wpływu na sumę długości relacji. Z~tego powodu przewidywania modelu moskiewskiego co do występowania częstości zmian są identyczne, jak w przypadku modelu stanfordzkiego.