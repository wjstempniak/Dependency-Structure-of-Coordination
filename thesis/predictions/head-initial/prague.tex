\subsubsection{Podejście praskie}

\begin{table}[h]
\centering
\begin{tabular}{lclc}

(L-L) & 
\begin{dependency}[hide label, edge unit distance=0.5ex, baseline=-\the\dimexpr\fontdimen22\textfont2\relax]
        \begin{deptext}
        $\odot$\&$\square$\&$\square$\&$\square$\&$\boxdot$\&$\square$\&$\square$\&$\square$\&$\square$\&$\square$\&$\square$\\
            \end{deptext}
	  \depedge{1}{5}{}
	  \depedge{5}{2}{}
	  \depedge{5}{6}{}
            \wordgroup{1}{2}{4}{L}
            \wordgroup{1}{6}{11}{R}
        \end{dependency}

& (L-R) &

   \begin{dependency}[hide label, edge unit distance=0.5ex, baseline=-\the\dimexpr\fontdimen22\textfont2\relax]
        \begin{deptext}
        $\odot$\&$\square$\&$\square$\&$\square$\&$\square$\&$\square$\&$\square$\&$\boxdot$\&$\square$\&$\square$\&$\square$\\
            \end{deptext}
	  \depedge{1}{8}{}
	  \depedge{8}{2}{}
	  \depedge{8}{9}{}
            \wordgroup{1}{2}{7}{L}
            \wordgroup{1}{9}{11}{R}
        \end{dependency}
        
\\ (0-L) &

\begin{dependency}[hide label, edge unit distance=0.5ex, baseline=-\the\dimexpr\fontdimen22\textfont2\relax]
        \begin{deptext}
        $\square$\&$\square$\&$\square$\&$\boxdot$\&$\square$\&$\square$\&$\square$\&$\square$\&$\square$\&$\square$\\
            \end{deptext}
	  \depedge{4}{1}{}
	  \depedge{4}{5}{}
            \wordgroup{1}{1}{3}{L}
            \wordgroup{1}{5}{10}{R}
        \end{dependency}
        
& (0-R) &

\begin{dependency}[hide label, edge unit distance=0.5ex, baseline=-\the\dimexpr\fontdimen22\textfont2\relax]
        \begin{deptext}
        $\square$\&$\square$\&$\square$\&$\square$\&$\square$\&$\square$\&$\boxdot$\&$\square$\&$\square$\&$\square$\\
            \end{deptext}
	  \depedge{7}{1}{}
	  \depedge{7}{8}{}
            \wordgroup{1}{1}{6}{L}
            \wordgroup{1}{8}{10}{R}
        \end{dependency}

\\ (R-L) &

\begin{dependency}[hide label, edge unit distance=0.5ex, baseline=-\the\dimexpr\fontdimen22\textfont2\relax]
        \begin{deptext}
        $\square$\&$\square$\&$\square$\&$\boxdot$\&$\square$\&$\square$\&$\square$\&$\square$\&$\square$\&$\square$\&$\odot$\\
            \end{deptext}
	  \depedge{11}{4}{}
	  \depedge{4}{1}{}
	  \depedge{4}{5}{}
            \wordgroup{1}{1}{3}{L}
            \wordgroup{1}{5}{10}{R}
        \end{dependency}
        
& (R-R) &

\begin{dependency}[hide label, edge unit distance=0.5ex,  baseline=-\the\dimexpr\fontdimen22\textfont2\relax]
        \begin{deptext}
        $\square$\&$\square$\&$\square$\&$\square$\&$\square$\&$\square$\&$\boxdot$\&$\square$\&$\square$\&$\square$\&$\odot$\\
            \end{deptext}
	  \depedge{11}{7}{}
	  \depedge{7}{1}{}
	  \depedge{7}{8}{}
            \wordgroup{1}{1}{6}{L}
            \wordgroup{1}{8}{10}{R}
        \end{dependency}
        
\\
\end{tabular}
\end{table}

W celu predykcji tendencji do umieszczania krótszego członu koordynacji należy policzyć sumę długości relacji zależnościowych. Ze schematu wynika, że:

\begin{itemize}
\item w~zdaniach (L-L) suma długości relacji jest \textbf{mniejsza}, niż w~zdaniach (L-R);
\item w~zdaniach (0-L) suma długości relacji jest \textbf{mniejsza}, niż w~zdaniach (0-R);
\item w~zdaniach (R-L) suma długości relacji jest \textbf{taka sama} jak w~zdaniach (R-R).
\end{itemize}

Powyższe różnice rosną wraz ze~wzrostem różnicy długości członów. Ze~względu na efekt DLM, użytkownicy języka są skłonni tworzyć zdania o krótszej łącznej długości członów tym częściej, im bardziej mają możliwość skrócić łączną długość relacji. Oznacza to, że wielkość różnicy długości członów koordynacji przekłada się bezpośrednio na częstość występowania zdań z krótszym lewym członem.

Na tej podstawie \cite{przepiorkowski2023conjunct} wyciągają wniosek, że model praski przewiduje, że wraz~ze wzrostem różnicy długości członów:
\begin{itemize}
\item odsetek koordynacji (L-L) względem wszystkich koordynacji (L) \textbf{rośnie};
\item odsetek koordynacji (0-L) względem wszystkich koordynacji (0) \textbf{rośnie};
\item odsetek koordynacji (R-L) względem wszystkich koordynacji (R) \textbf{nie zmienia się}.
\end{itemize}

W następnych punktach analogiczne rozumowania dla pozostałych podejść przedstawione są w sposób skrócony.

