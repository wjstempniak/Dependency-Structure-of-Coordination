\paragraph{Odwrócone podejście moskiewskie (E)}

Model przewiduje następujące długości relacji:

\begin{table}[H]
\begin{tabular}{lcllcl}

(L-L) &

\begin{dependency}[hide label, edge unit distance=0.5ex, baseline=-\the\dimexpr\fontdimen22\textfont2\relax]
        \begin{deptext}
        $\odot$\&a\&$\square$\&$\boxdot$\&a+b\&$\square$\\
        \end{deptext}
		\depedge{1}{6}{}
		\depedge{6}{4}{}
		\depedge{4}{3}{}
        \wordgroup{1}{2}{3}{L}
        \wordgroup{1}{5}{6}{R}
        \end{dependency}

& $S=2a+2b$ & 

(L-R) &

\begin{dependency}[hide label, edge unit distance=0.5ex, baseline=-\the\dimexpr\fontdimen22\textfont2\relax]
        \begin{deptext}
        $\odot$\&a+b\&$\square$\&$\boxdot$\&a\&$\square$\\
        \end{deptext}
		\depedge{1}{3}{}
		\depedge{1}{6}{}
		\depedge{3}{4}{}
		\wordgroup{1}{2}{3}{L}
		\wordgroup{1}{5}{6}{R}
        \end{dependency}
        
& $S=2a+b$ \\ 

(0-L) &

\begin{dependency}[hide label, edge unit distance=0.5ex, baseline=-\the\dimexpr\fontdimen22\textfont2\relax]
        \begin{deptext}
        a\&$\square$\&$\boxdot$\&a+b\&$\square$\\
        \end{deptext}
		\depedge{5}{3}{}
		\depedge{3}{2}{}
        \wordgroup{1}{1}{2}{L}
        \wordgroup{1}{4}{5}{R}
        \end{dependency}

& $S=a+b$ & 

(0-R) &

\begin{dependency}[hide label, edge unit distance=0.5ex, baseline=-\the\dimexpr\fontdimen22\textfont2\relax]
        \begin{deptext}
        a+b\&$\square$\&$\boxdot$\&a\&$\square$\\
        \end{deptext}
		\depedge{5}{3}{}
		\depedge{3}{2}{}
        \wordgroup{1}{1}{2}{L}
        \wordgroup{1}{4}{5}{R}
        \end{dependency}
        
& $S=a$ \\

(R-L) &

\begin{dependency}[hide label,edge unit distance=0.5ex, baseline=-\the\dimexpr\fontdimen22\textfont2\relax]
        \begin{deptext}
        a\&$\square$\&$\boxdot$\&a+b\&$\square$\&$\odot$\\
        \end{deptext}
		\depedge{6}{5}{}
		\depedge{5}{3}{}
		\depedge{3}{2}{}
		\wordgroup{1}{1}{2}{L}
		\wordgroup{1}{4}{5}{R}
        \end{dependency}
        
& $S=a+b$ &

(R-R) &

\begin{dependency}[hide label, edge unit distance=0.5ex, baseline=-\the\dimexpr\fontdimen22\textfont2\relax]
        \begin{deptext}
           a+b\&$\square$\&$\boxdot$\&a\&$\square$\&$\odot$\\
        \end{deptext}
		\depedge{6}{5}{}
		\depedge{5}{3}{}
		\depedge{3}{2}{}
        \wordgroup{1}{1}{2}{L}
        \wordgroup{1}{4}{5}{R}
        \end{dependency}

& $S=a$ \\

\end{tabular}
\end{table}

Wraz~ze wzrostem różnicy długości członów (\emph{b}):
\begin{itemize}
\item odsetek koordynacji (L-L) względem wszystkich koordynacji (L) \textbf{spada};
\item odsetek koordynacji (0-L) względem wszystkich koordynacji (0) \textbf{spada};
\item odsetek koordynacji (R-L) względem wszystkich koordynacji (R) \textbf{spada}.
\end{itemize}