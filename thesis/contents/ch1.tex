\chapter{Wprowadzenie} \label{ch1}
\section{Cel pracy}

Jedno z~pytań, na które próbuje odpowiedzieć teoria składni, brzmi: ,,Dlaczego układamy słowa w~zdaniach w~ten sposób, a nie inaczej?''

Rozważmy dwa zdania:

\begin{exe}
\ex \label{gazeta+książka} Wczoraj \emph{czytałem} [[gazetę] i~[bardzo ciekawą książkę przygodową]].
\ex \label{książka+gazeta} Wczoraj \emph{czytałem} [[bardzo ciekawą książkę przygodową] i~[gazetę]].
\end{exe}

W~zdaniach \eqref{gazeta+książka} i~\eqref{książka+gazeta} znajdują się koordynacje zawierające dwa człony: \emph{gazetę} oraz \emph{bardzo ciekawą książkę przygodową}. Jedyną różnicą między tymi zdaniami jest kolejność ustawienia tych członów. Człony mają podobne własności syntaktyczne i~semantyczne. W~związku z~tym można by przypuszczać, że nie ma znaczenia, który człon zostanie umieszczony w~zdaniu jako pierwszy. Niemniej jednak człony różnią się długością. Długość członu można liczyć na rozmaite sposoby -- w~słowach, tokenach, sylabach lub znakach. Na przykład, \emph{bardzo ciekawą książkę przygodową} ma 4~tokeny i~33 znaki, zaś \emph{gazetę} ma 1 token i~6 znaków. 

Pierwszą rzeczą, jaką wykazuję, jest to, że długość członów koordynacji jest istotnym czynnikiem w~wyborze ich kolejności. Wynika to ze zjawiska znanego jako Dependency Length Minimization (DLM). Jest to uniwersalna zależność występująca w~języku naturalnym. Polega ona na tym, że słowa układane są w~zdania w~sposób minimalizujący sumę długości relacji składniowych między nimi \citep{temperley2007minimization, futrell2015large}. Ponieważ długość relacji wewnątrz członów koordynacji jest stała, jedynie ustawienie członów w~zdaniu ma wpływ na sumę długości relacji. 
W~tej pracy badam, w~jaki sposób pozycja nadrzędnika wpływa na ustawienie kolejności członów. Pokazuję, że~w~językach inicjalnych, takich jak polski czy angielski, zdanie o strukturze takiej jak \eqref{gazeta+książka} ma większą szansę na pojawienie się w~języku naturalnym niż zdanie takie, jak \eqref{książka+gazeta}.

Drugim problemem, jaki podejmuję, jest kwestia opisu relacji składniowych. Wśród lingwistów nie ma zgody co do tego, jakie dokładnie relacje zależnościowe łączą poszczególne elementy koordynacji. Istnieją cztery uznawane sposoby opisu -- model praski, londyński, stanfordzki i~moskiewski \citep{popel2013coordination, przepiorkowski2023conjunct}. Niemniej jednak, zakładając prawdziwość DLM, można wykazać, że nie wszystkie podejścia są poprawne. W~tej pracy sprawdzam, które modele tworzą drzewa zależnościowe zgodne z~przewidywaniami DLM.

\cite{przepiorkowski2023conjunct} przeprowadzili analizę koordynacji w~języku angielskim. Wykazali, że modele stanfordzki i~moskiewski nie są poprawnymi metodami opisu koordynacji. Moim głównym celem jest replikacja tego badania oraz rozszerzenie go na trzynaście języków. W~ramach badania analizuję koordynacje osobno w~językach inicjalnych oraz finalnych. Wykazuję, że w~przypadku tych pierwszych krótszy jest zwykle lewy człon, zaś w~przypadku tych drugich -- prawy człon.

%WYNIKI W SKRÓCIE. Napisać o UD

W badaniu wykorzystuję korpusy zależnościowe opisane w wersji 2.14 standardu Universal Dependencies \citep{de2021universal}. Korpusy te pochodzą ze strony \url{https://universaldependencies.org/}. 
Wszystkie wykorzystane dane oraz skrypty użyte w analizie dostępne są w~repozytorium pod adresem \url{https://github.com/wjstempniak/Dependency-Structure-of-Coordination}.

\section{Struktura pracy}

Niniejszy pierwszy rozdział pracy poświęcam wprowadzeniu do tematu i~opisaniu struktury pracy. W~rozdziale~\ref{ch2}. przedstawiam podstawy teoretyczne mojej pracy. Przedstawiam dokładnie problem opisywania struktury koordynacji i~omawiam standard Universal Dependencies~(UD). W~rozdziale~\ref{ch3}. przedstawiam przewidywania różnych modeli dotyczących struktury koordynacji w~językach inicjalnych i~finalnych. W~rozdziale~\ref{ch4}. przedstawiam kolejne etapy przeprowadzonego przeze mnie badania. Szczególną uwagę poświęcam problemowi automatycznego wyznaczania granic członów koordynacji. W~rozdziale~\ref{ch5}. opisuję analizę statystyczną uzyskanych przeze mnie wyników. W~rozdziale~\ref{ch6}. przedstawiam wnioski wyciągnięte z~badania. Pracę kończę rozdziałem~\ref{ch7}., w~którym opisuję ograniczenia mojej pracy oraz proponuję dalsze możliwości badań w~tej dziedzinie.
