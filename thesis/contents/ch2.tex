\chapter{Wstęp teoretyczny} \label{ch2}
\section{Zależności składniowe}

Zależności składniowe są centralnym zagadnieniem gramatyki zależnościowej, czyli dziedziny teorii składni zajmującej się połączeniami pomiędzy poszczególnymi słowami wchodzącymi w~skład zdania. Są to relacje łączące dwa elementy: nadrzędnik oraz zależny od niego podrzędnik. Elementami łączonymi przez zależności są tokeny -- przez to pojęcie należy rozumieć słowa oraz znaki interpunkcyjne, a także, według niektórych podejść, części niektórych słów.

Relacje występujące w~obrębie zdania można opisać za pomocą grafu zwanego drzewem zależnościowym. \eqref{róża-drzewo} jest przykładowym drzewem zależnościowym zdania \eqref{róża}:

\begin{exe}
\ex \label{róża} W~ręku trzymałam czerwoną różę.\\ 
\ex \label{róża-drzewo}
\begin{dependency}[theme = simple, baseline=-\the\dimexpr\fontdimen22\textfont2\relax]
\begin{deptext}[column sep=1em]
W~\& ręku \& trzymał \& am \& czerwoną \& różę \& .  \\ 
\end{deptext}
\depedge{1}{2}{}
\depedge{3}{1}{}
\deproot{3}{}
\depedge{3}{4}{}
\depedge{6}{5}{}
\depedge{3}{6}{}
\depedge{3}{7}{}
\end{dependency}
\end{exe}

Warto zauważyć, że słowo \emph{trzymałam} składa się z~dwóch tokenów -- \emph{trzymał} oraz~\emph{am}.

Należy odróżnić zależności składniowe od relacji składnikowych. Relacje składnikowe łączą poszczególne słowa we~frazy oraz frazy w~bardziej złożone frazy w~zdania \citep{chomsky1956three}.
Schemat \eqref{róża-frazy} przedstawia drzewo zależnościowe zdania \eqref{róża} z~wyróżnionymi niektórymi frazami.

\begin{exe}
\ex \label{róża-frazy}
\begin{dependency}[theme=simple, baseline=-\the\dimexpr\fontdimen22\textfont2\relax]
\begin{deptext}[column sep=1em]
\textbf{W} \& ręku \& trzymał \& am \& czerwoną \& \textbf{różę} \& .  \\ 
\end{deptext}
\depedge{1}{2}{}
\depedge{3}{1}{}
\deproot{3}{}
\depedge{3}{4}{}
\depedge{6}{5}{}
\depedge{3}{6}{}
\depedge{3}{7}{}
\wordgroup{1}{1}{2}{P1}
\wordgroup{1}{5}{6}{P2}
\end{dependency}
\end{exe}

Słowo określające kategorię syntaktyczną danej frazy nazywane jest głową tej frazy \citep{hoeksema1992head}. Przykładowo, głową frazy przyimkowej \emph{W~ręku} jest przyimek \emph{W}, zaś głową frazy rzeczownikowej \emph{czerwoną różę} -- rzeczownik \emph{różę}. 

W~drzewie zależnościowym głowę danej frazy można rozpoznać po tym, że nie posiada nadrzędników w~jej obrębie.

\section{Universal Dependencies}

Obecnie nie ma w~lingwistyce jednej, szeroko akceptowanej formuły opisu relacji zależnościowych. Standardem, z~którego korzystam, jest Universal Dependencies. Jest to formalizm tworzący ,,ramy dla spójnego opisu gramatyki'' \citep{de2021universal} języków, a więc takich cech jak części mowy, cechy morfologiczne czy właśnie relacje składniowe. 

Universal Dependencies opisuje zależności składniowe za pomocą zestawu uniwersalnych znaczników, charakterystycznych dla rodzaju relacji. Na przykład, relację łączącą podmiot nominalny z~orzeczeniem opisuje znacznik \emph{nsubj}, a dopełnienie bliższe -- \emph{obj}. Znaczniki UD posiadają różne warianty. Podmiot zdania w~trybie biernym opisuje znacznik \emph{nsubj:pass}. W~tej pracy traktuję każdy wariant jako osobny znacznik, chyba że zaznaczam inaczej.

Poniżej znajduje się drzewo zależnościowe zdania \eqref{róża} opisane według standardu~UD.\footnote{Drzewa zależnościowe w~formacie UD wszystkich zdań niepochodzących z~korpusów zostały stworzone przez automatyczne parsowanie parserem Trankit \citep{van2021trankit}.}

\begin{exe}
\ex \label{róża-UD}
\begin{dependency}[baseline=-\the\dimexpr\fontdimen22\textfont2\relax]
\begin{deptext}[column sep=1em]
W~\& ręku \& trzymał \& am \& czerwoną \& różę \& .  \\ 
\end{deptext}
\depedge{2}{1}{case}
\depedge{3}{2}{obl}
\deproot{3}{root}
\depedge{3}{4}{aux:clitic}
\depedge{6}{5}{amod}
\depedge{3}{6}{obj}
\depedge{3}{7}{punct}
\end{dependency}
\end{exe}

Należy zwrócić uwagę na krawędzie łączące tokeny \emph{W}, \emph{ręku} oraz \emph{trzymał}. Według Universal Dependencies głową frazy przyimkowej \emph{W~ręku} jest rzeczownik \emph{ręku}. Jest to sprzeczne z~teorią lingwistyczną i~wskazuje na niedoskonałość tego standardu. Niemniej jednak w~niniejszej pracy będę zakładał poprawność UD.

\section{Koordynacja}

Koordynacja lub konstrukcja współrzędnie złożona to zestawienie w~zdaniu dwóch lub więcej elementów pełniących tę samą funkcję syntaktyczną. Elementy te nazywa się członami koordynacji.

Problem wyznaczania granic członów jest jednym z~ważniejszych aspektów analizy koordynacji. Nie jest to zadanie trywialne i~często wymaga odwołania się do semantyki. Rozważmy następujące przykłady:

\begin{exe}
\ex \label{dzieci+nauczyciele} Niesforne dzieci i~nauczyciele pojechali na wycieczkę.
\ex \label{uczniowie+nauczyciele} Niesforni uczniowie i~nauczyciele pojechali na wycieczkę.
\end{exe}

W~przypadku zdania \eqref{dzieci+nauczyciele} słowo \emph{Niesforne} jest niekontrowersyjnym podrzędnikiem słowa \emph{dzieci}, ponieważ wynika to ze związku zgody.  Z~tego wynika, że granice członów koordynacji wyglądają następująco:
% poprawić: powt. wynika

\begin{exe}
\ex \label{dzieci+nauczyciele-nawiasy}
{[[Niesforne dzieci] i~[nauczyciele]] \emph{pojechali} na wycieczkę.}
\end{exe}

W~zdaniu \eqref{uczniowie+nauczyciele} \emph{Niesforni} może opisywać zarówno \emph{uczniowie}, jak i~\emph{nauczyciele}. W~takim wypadku syntaktyka dopuszcza dwie interpretacje zdania:

\begin{exe}
\ex \label{uczniowie+nauczyciele-dobrze}
{[[Niesforni uczniowie] i~[nauczyciele]] \emph{pojechali} na wycieczkę.}
\ex \label{uczniowie+nauczyciele-źle}
{[Niesforni [[uczniowie] i~[nauczyciele]]] \emph{pojechali} na wycieczkę.}
\end{exe}

W~tej sytuacji prawdopodobne granice członów można wyznaczyć jedynie odwołując się do znaczeń poszczególnych słów. Zdroworozsądkowa semantyka nakazuje nam przyjąć interpretację \eqref{uczniowie+nauczyciele-dobrze}. Poniżej znajduje się drzewo zależnościowe zdania \eqref{uczniowie+nauczyciele} w~interpretacji \eqref{uczniowie+nauczyciele-dobrze}, opisane w~standardzie UD, z~zaznaczonymi istotnymi elementami koordynacji.

\begin{exe}
\ex \label{uczniowie+nauczyciele-UD}
\resizebox{\linewidth}{!}{
\begin{dependency}[baseline=-\the\dimexpr\fontdimen22\textfont2\relax]
\begin{deptext}[column sep=.5cm]
Niesforni  \& \textbf{uczniowie} \& i~\& \textbf{nauczyciele} \& \emph{pojechali} \& na \& wycieczkę \& .\\
\end{deptext}
\deproot{5}{root}
\depedge{2}{1}{amod}
\depedge{5}{2}{nsubj}
\depedge{2}{4}{conj}
\depedge{4}{3}{cc}
\depedge{5}{8}{punct}
\depedge{5}{7}{obl}
\depedge{7}{6}{case}
\wordgroup{1}{1}{2}{L}
\wordgroup{1}{4}{4}{R}
\end{dependency}
}
\end{exe}

Człony koordynacji często połączone są spójnikami. Przez spójnik koordynacji rozumiem właśnie ten spójnik, który łączy jej człony. Zakładam, że każda konstrukcja współrzędnie złożona ma co najwyżej jeden spójnik. W~przypadku koordynacji \eqref{uczniowie+nauczyciele-dobrze} jest to~\emph{i}.

Przez lewy i~prawy człon koordynacji rozumiem odpowiednio pierwszy i~ostatni człon występujący w~zdaniu (niezależnie od tego, ile jest tych członów) -- w~\eqref{uczniowie+nauczyciele-dobrze} są to \emph{Niesforne dzieci} oraz \emph{nauczyciele}.

Głową członu jest token, który nie ma nadrzędnika w~obrębie tego członu -- dla członu \emph{Niesforni uczniowie} będzie nim słowo \emph{uczniowie}.

Nadrzędnik koordynacji to token, który jest najbliższym wspólnym przodkiem wszystkich członów koordynacji i~jej spójnika. W~koordynacji \eqref{uczniowie+nauczyciele-dobrze} będzie to słowo \emph{pojechali}.

Ostatnią ważną charakterystyką konstrukcji współrzędnie złożonej jest pozycja nadrzędnika -- parametr określający umiejscowienie nadrzędnika koordynacji względem jej członów. Tabela \ref{pozycja-nadrzędnika} przedstawia możliwe pozycje nadrzędnika wraz z~przykładowymi zdaniami.

\begin{table}[!h]
\centering
\begin{tabular}{l l l}
\toprule
ozn.	& pozycja			& przykład zdania								\\
\midrule
(L) & po lewej stronie	& Drzewo \textit{sadzą} [[Pat] i~[Mat]].		\\
(0) & brak nadrzędnika	& [[Posadzili] i~[podlali]] drzewo.				\\
(R) & po prawej stronie	& [[Pat] i~[Mat]] \textit{posadzili} drzewo. 			\\
(M) & po środku			& [[Pat] -- \textit{powtórzyłem} -- oraz [Mat]].	\\
\bottomrule
\end{tabular}
\caption{Pozycja nadrzędnika}
\label{pozycja-nadrzędnika}
\end{table}

Konstrukcje współrzędnie złożone z~nadrzędnikiem po środku są bardzo rzadkie i~nie występują w~wielu językach (m.in. w~języku angielskim). Oznacza to, że uzyskanie istotnych statystycznie wyników dla koordynacji (M) jest bardzo trudne. Ponadto, wstępna analiza koordynacji typu (M) w~języku polskim wykazała, że mniej niż 20\% z~nich zostało opisane w~sposób prawidłowy. W~związku z~tym w~niniejszej pracy biorę pod uwagę wyłącznie koordynacje typu (L), (0) i~(R).

\section{Dependency Length Minimization}

Szyk zdania ma wiele ograniczeń. Najlepiej opisane są te dotyczące sposobu ustawienia najważniejszych części zdania, takich jak podmiot, orzeczenie i~dopełnienia. Nie mają one większego wpływu na ustawienie innych elementów zdania, takich jak między innymi umiejscowienie członów koordynacji.

W~tej pracy skupiam się na zjawisku, które zdaniem wielu lingwistów istotnie wpływa na sposób układania słów w~zdaniu -- Dependency Length Minimization (DLM). W~języku występuje naturalna tendencja do jak największego skracania sumy długości relacji zależnościowych. Innymi słowy, słowa układane są w~takiej kolejności, żeby każde dwa słowa połączone bezpośrednio relacją składniową stały możliwie blisko siebie \citep{temperley2007minimization}.

Wyjaśnienie przyczyn tego zjawiska jest dość intuicyjne. \cite{king1991individual} pokazują, że podczas składania i~odkodowywania zdań w~umysłach użytkowników powstają reprezentacje relacji składniowych. Utrzymywanie tych reprezentacji w~czasie wykorzystuje pamięć roboczą \citep[s. 596]{king1991individual}. Im dłuższa jest relacja, tym dłużej jej reprezentacja jest utrzymywana w~pamięci. Zbyt długo utrzymywane reprezentacje powodują błędy w~konstrukcji i~rozumieniu zdań. W~związku z~tym, w~celu oszczędzania pamięci roboczej i~zmniejszania liczby błędów, należy przechowywać aktywne reprezentacje zależności składniowych jak najkrócej, a co za tym idzie, minimalizować długość zależności.

W badaniach nad DLM używane są rozmaite sposoby mierzenia długości zależności. Do najpopularniejszych jednostek miary należą morfemy, sylaby i~fonemy \citep{lohmann2014english}. \cite{przepiorkowski2023conjunct} w~swoim badaniu mierzą długość członów w~słowach, sylabach i~znakach. W celu replikacji badań w~niniejszej analizie stosuję identyczną metodologię.

\cite{lohmann2014english} wskazuje, że najlepszą metodą pomiaru długości członu jest prawdopodobnie złożoność syntaktyczna. Należy przez to rozumieć liczbę węzłów w~relacjach składniowych łączących słowa wchodzące w~skład członu. Większa złożoność syntaktyczna przekłada się bezpośrednio na większe zaangażowanie pamięci roboczej w~przetwarzanie języka. 

Spośród używanych przeze mnie miar najlepszym estymatorem złożoności syntaktycznej frazy jest liczba słów. W związku z~tym, na potrzeby dyskusji wyników mojej pracy uznaję liczbę słów w~członie za najpewniejszą miarę długości członu konstrukcji współrzędnie złożonej.

Uniwersalne występowanie DLM w~językach naturalnych jest potwierdzone w~badaniach \citep{futrell2015large}. W~tej pracy zakładam, że jest to istotny czynnik mający także wpływ na ustawienie członów koordynacji. 

\section{Kolejność członów koordynacji} \label{kolejność}

DLM nie jest jedynym czynnikiem wpływającym na kolejność ustawiania członów koordynacji.
\cite{lohmann2014english} opisuje wiele innych zjawisk, które to robią. W niniejszym punkcie omawiam najważniejsze z~nich i~opisuję ich możliwe interakcje z~DLM.

\paragraph{Konwencja lub następstwo czasowe}

W~przypadku zdania \eqref{P+P} kolejność doboru członów wynika z~konwencji, natomiast zdania \eqref{po-bożemu} i~\eqref{nie-po-bożemu} mają zupełnie inne znaczenie. W~przykładach takich jak powyższe nie można więc mówić o determinowaniu ustawienia kolejności członów przez DLM. 

\begin{exe}
\ex \label{P+P} {[[Panie] i~[Panowie]]!}
\ex \label{po-bożemu} {[[Pobrali się] i~[mieli dziecko]].} % znaleźć źródło tego zdania
\ex \label{nie-po-bożemu} {[[Mieli dziecko] i~[pobrali się]].}
\end{exe}

Te tendencje działają jednak w~dwie strony (w tym samym oraz w~przeciwnym kierunku, co DLM). W~związku z~tym wychodzę z~założenia, że podczas analizy obszernych korpusów językowych ich łączny wpływ na kolejność członów koordynacji jest pomijalny.

\paragraph{Czynniki pragmatyczne} 

\cite{lohmann2014english} wymienia rozmaite czynniki pragmatyczne wpływające na ustawienie członów koordynacji. Wskazuje różne hipotezy, zgodnie z~którymi pierwszeństwo mają człony mówiące o obiektach bliższych lub lepiej znanych nadawcy wypowiedzi. 

Warto zwrócić uwagę na sytuacje, w~których koordynacja wprowadza do dyskursu nowe obiekty.

\begin{exe}
\ex \label{paweł+gaweł} {[[Paweł] i~[jego sąsiad Gaweł]]}
\ex \label{gaweł+paweł} {?? [Jego sąsiad Gaweł] i~[Paweł]]}
\end{exe}

W przykładzie \eqref{paweł+gaweł} treść członu \textit{[Paweł]} stanowi punkt odniesienia do wprowadzenia kolejnego obiektu, czyli treści członu \textit{[jego sąsiad Gaweł]}. W takiej sytuacji ustawienie członów w~odwrotnej kolejności jest niepoprawne pragmatycznie (zob. \eqref{gaweł+paweł}). Punkty odniesienia, takie jak \textit{Paweł}, są co do zasady opisywane w~krótszy sposób niż nowe obiekty, takie jak \textit{Gaweł}. Oznacza to, że w~takich sytuacjach krótszy człon występuje po lewej stronie koordynacji niezależnie od DLM. Niemniej jednak, jest to sytuacja występująca relatywnie rzadko.

\paragraph{Częstość używania słowa}
\cite{fenk1989word} stawia hipotezę, że człony składające się z~częściej występujących w~języku słów  częściej pojawiają się jako pierwsze w~konstrukcjach współrzędnie złożonych. Ponieważ w~języku naturalnym krótsze słowa występują co do zasady częściej niż długie \citep{zipf1946psychology}, to zjawisko może powodować ustawianie krótszego członu na początku koordynacji \citep[s. 54]{lohmann2014english}.

Zgodnie z~tą hipotezą, koordynacja \eqref{pies+hipopotam} ma większą szansę na pojawienie się w~języku niż \eqref{hipopotam+pies}, ponieważ w~języku polskim słowo \textit{pies} występuje częściej niż \textit{hipopotam} \citep{przepiorkowski2012narodowy}.

\begin{exe}
\ex \label{pies+hipopotam} {[[Pies] i~[hipopotam]]}
\ex \label{hipopotam+pies} {[[Hipopotam] i~[pies]]}
\end{exe}

\paragraph{Prozodia i~akcent}

Kolejnym czynnikiem mającym duży wpływ na ustawienie członów koordynacji jest prozodia. \cite{lohmann2014english} wskazuje, że koordynacje o krótkich członach są konstruowane tak, żeby akcentowane sylaby tworzyły rytm.

\begin{exe}
\ex \label{jan+maria} {
\metrics{   _   u |   _   u  }
        {[[Jan] i | [Ma-ria]]}
}
\ex \label{maria+jan} {
\metrics{   _   u  u   _   }
        {[[Ma-ria] i [Jan]]}}
\end{exe}

W przykładzie \eqref{jan+maria} sylaby nieakcentowane i~akcentowane występują naprzemiennie. Dzięki temu zdanie zawierające taką koordynację zawiera rytm oparty o stopy metryczne\footnote{
Na schemacie \metricsymbols{u} oznacza sylabę krótką (nieakcentowaną), zaś \metricsymbols{_} sylabę długą (akcentowaną). Jeśli w~zdaniu można wydzielić stopy, są one rozdzielone znakiem \metricsymbols{|}.  Sekwencje \metricsymbols{| u _ |} oraz \metricsymbols{| _  u |} tworzą odpowiednio jamb i~trochej -- dwie najprostsze stopy metryczne.}.
Natomiast w~przykładzie \eqref{maria+jan} sylaby nie tworzą sekwencji rytmicznej. Powoduje to, że konstrukcja \eqref{jan+maria} ma większe szanse pojawić się w~języku, niż \eqref{maria+jan} (\citealt{mcdonald1993word}, \citealt{wright2005ladies}).

Na ostatnie dwa czynniki (częstość występowania słowa oraz rozkład sylab akcentowanych) znaczący wpływ ma długość członów. Dotyczą one głównie koordynacji krótkich, o członach jedno- lub dwuwyrazowych i~o niewielkich różnicach długości członów. Oznacza to, że mogą one w~tych przypadkach wzmacniać lub osłabiać efekt DLM. Z tego powodu w~niniejszym badaniu analizuję również koordynacje dłuższe i~takie, w~których różnica długości członów jest znaczna.

Skutki istnienia innych czynników mających wpływ na ustawienie członów koorynacji opisuję dokładniej w~rozdziale \ref{ch7}. 

\section{Języki inicjalne oraz finalne}

Jedną z~cech, według których można sklasyfikować języki naturalne, jest generalna pozycja nadrzędnika względem podrzędnika. Ze względu na ten parametr w~językoznawstwie wyróżnia się dwa rodzaje języków: inicjalne (ang. \emph{head-initial}) oraz finalne (ang. \emph{head-final}). Klasyfikacja ta oparta jest na cechach struktur gramatycznych występujących w~danym języku, w~tym przede wszystkim na szyku zdaniowym \citep{polinsky2020headedness}.

W~językach inicjalnych występuje generalna tendencja do umieszczania nadrzędnika przed podrzędnikiem. Do tej kategorii należy wiele języków indoeuropejskich, między innymi języki angielski, francuski, hiszpański, grecki oraz polski. W~językach finalnych natomiast nadrzędnik zwykle umieszczany jest za podrzędnikiem. Przykładami takich języków są japoński, koreański oraz turecki \citep{polinsky2012headedness}.

Podział ten nie jest zero-jedynkowy. Istnieje wiele przykładów granicznych oraz takich, w~których przypadku tendencja jest słaba. Na przykład język niemiecki jest zasadniczo finalny, jednak występuje w~nim wiele struktur gramatycznych nadających mu cechy języka inicjalnego \citep{polinsky2012headedness}. W niniejszej pracy języki nienależące ściśle do żadnej z~tych dwóch kategorii nazywam językami mieszanymi.

Poniżej znajdują się drzewa zależnościowe przykładowego zdania w~języku typowo inicjalnym (angielskim) oraz finalnym (tureckim). W~\eqref{inicjalne} widoczna jest tendencja do umieszczania głów członów po lewej, zaś w~\eqref{finalne} po prawej stronie zdania.

\begin{exe}
\ex \label{inicjalne}
Everyone has the right to a nationality.\footnote{Powszechna Deklaracja Praw Człowieka, art. 15, ust. 1. \url{https://www.un.org/en/about-us/universal-declaration-of-human-rights}}

\resizebox{\linewidth}{!}{
\begin{dependency}[baseline=-\the\dimexpr\fontdimen22\textfont2\relax]
\begin{deptext}[column sep=1em, row sep=.1ex]
Everyone \& has \& the \& right \& to \& a \& nationality \& .  \\
każdy \& mieć\textsc{.3sg.prs} \& \textsc{art} \& prawo \& do \& \textsc{art} \& obywatelstwo \&  \\ 
\end{deptext}
\depedge{2}{1}{nsubj}
\deproot{2}{root}
\depedge{4}{3}{det}
\depedge{2}{4}{obj}
\depedge{7}{5}{case}
\depedge{7}{6}{det}
\depedge[edge unit distance=2.8ex]{4}{7}{nmod}
\depedge[edge unit distance=1.7ex]{2}{8}{punct}
\end{dependency}
}
,,Każdy człowiek ma prawo do posiadania obywatelstwa.''

\ex \label{finalne}
Her ferdin bir uyrukluk hakkı vardır.\footnote{Tłumaczenie pochodzi ze~strony \url{https://www.ohchr.org/en/human-rights/universal-declaration/translations/turkish-turkce}. Słowa w~języku tureckim zostały zanalizowane przy użyciu TRmorph \citep{coltekin2010freely} oraz oznakowane na podstawie reguł określonych w~pracach \cite{haspelmath2014leipzig} i~\cite{bedir2021overcoming}. Dziękuję Berkemu \c{S}en\c{s}ekerci za sprawdzenie glos.}

\resizebox{\linewidth}{!}{
\begin{dependency}[baseline=-\the\dimexpr\fontdimen22\textfont2\relax]
\begin{deptext}[column sep=1em]
Her \& ferd-in \& bir \& uyruk-luk \& hak-kı \& var-dır \& .  \\ 
\textsc{det} \& osoba\textsc{-gen} \& \textsc{det} \& obywatel-stwo \& prawo\textsc{-3sg.poss} \& być-\textsc{cop.prs} \&  \\
\end{deptext}
\depedge{2}{1}{det}
\depedge{5}{2}{nmod:poss}
\depedge{4}{3}{det}
\depedge{5}{4}{nmod:poss}
\depedge{6}{5}{nsubj}
\deproot{6}{root}
\depedge{6}{7}{cop}
\depedge{6}{8}{punct}
\end{dependency}
}

,,Każdy człowiek ma prawo do posiadania obywatelstwa.''
\end{exe}

\section{Struktura zależnościowa koordynacji}

W~językoznawstwie nie ma pełnej zgody co do tego, jak opisywać relacje zależnościowe. Kontrowersje dotyczą nie tylko nazw i~rodzajów zależności (czyli etykiet na krawędziach), lecz także struktury zależnościowej struktur gramatycznych (czyli tego, które tokeny są połączone z~którymi w~ramach danej struktury składniowej).

Istnieją cztery główne podejścia do opisu struktury konstrukcji współrzędnie złożonych. Przedstawiają je poniższe schematy, w~których $\odot$ oznacza nadrzędnik koordynacji, $\boxdot$ spójnik, $\square$ pozostałe tokeny, zaś duże prostokąty symbolizują granice członów koordynacji. Relacje wewnątrz członów koordynacji nie są zaznaczone\footnote{
Poniższe schematy oparte są na schematach  z~pracy \cite{przepiorkowski2023conjunct}.}.

\begin{table}[H]

\centering

\begin{tabular}{c c}

\textbf{Spójnikowe/Praskie}

&

\textbf{Wielogłowe/Londyńskie}

\\

\begin{dependency}[hide label, edge unit distance=0.5ex]

        \begin{deptext}
        $\odot$\&$\square$\&$\square$\&$\square$\&,\&$\square$\&$\square$\&$\square$\&$\boxdot$\&$\square$\&$\square$\&$\square$\\
            \end{deptext}
            \depedge{1}{9}{}
            \depedge{9}{2}{}
            \depedge{9}{6}{}
            \depedge{9}{10}{}
            \wordgroup{1}{2}{4}{c1}
            \wordgroup{1}{6}{8}{c2}
            \wordgroup{1}{10}{12}{c3}
        \end{dependency}

&

\begin{dependency}[hide label, edge unit distance=0.5ex]

        \begin{deptext}
        $\odot$\&$\square$\&$\square$\&$\square$\&,\&$\square$\&$\square$\&$\square$\&$\boxdot$\&$\square$\&$\square$\&$\square$\\
            \end{deptext}
            \depedge{1}{2}{}
            \depedge{1}{6}{}
            \depedge{1}{10}{}
            \depedge{10}{9}{}
            \wordgroup{1}{2}{4}{c1}
            \wordgroup{1}{6}{8}{c2}
            \wordgroup{1}{10}{12}{c3}
        \end{dependency}

\vspace{.5cm}
\\ 

\textbf{Bukietowe/Stanfordzkie}

&

\textbf{Łańcuchowe/Moskiewskie}

\\

\begin{dependency}[hide label, edge unit distance=0.5ex]

        \begin{deptext}
        $\odot$\&$\square$\&$\square$\&$\square$\&,\&$\square$\&$\square$\&$\square$\&$\boxdot$\&$\square$\&$\square$\&$\square$\\
            \end{deptext}
            \depedge{1}{2}{}
            \depedge{2}{6}{}
            \depedge{2}{10}{}
            \depedge{10}{9}{}
            \wordgroup{1}{2}{4}{c1}
            \wordgroup{1}{6}{8}{c2}
            \wordgroup{1}{10}{12}{c3}
        \end{dependency}

&

\begin{dependency}[hide label, edge unit distance=0.5ex]

        \begin{deptext}
        $\odot$\&$\square$\&$\square$\&$\square$\&,\&$\square$\&$\square$\&$\square$\&$\boxdot$\&$\square$\&$\square$\&$\square$\\
            \end{deptext}
            \depedge{1}{2}{}
            \depedge{2}{6}{}
            \depedge{6}{9}{}
            \depedge{9}{10}{}
            \wordgroup{1}{2}{4}{c1}
            \wordgroup{1}{6}{8}{c2}
            \wordgroup{1}{10}{12}{c3}
        \end{dependency}

\end{tabular}
\end{table}


\cite{przepiorkowski2023conjunct} pokazują, że jedynie podejścia praskie oraz londyńskie mogą poprawnie opisywać strukturę zależnościową koordynacji przy założeniu o poprawności DLM. Wynika to z~dynamiki proporcji koordynacji z~krótszym lewym członem w~zależności od obecności i~pozycji nadrzędnika oraz z~założenia, że umieszczanie krótszego członu na początku koordynacji uległo gramatykalizacji. Rozumowanie to jest szczegółowo przedstawione w~rozdziale \ref{ch3}.

\cite{przepiorkowski2023conjunct} opierają swoją analizę na dwóch istotnych założeniach. Po pierwsze, autorzy zakładają, że gramatykalizacja umieszczania krótszego członie po lewej stronie w~języku angielskim wynika z~faktu, że lewe człony są istotnie częściej krótsze niż prawe. Po drugie, głowa członu co do zasady znajduje się na początku członu. 

W~niniejszej pracy pokazuję, że te założenia prawdziwe są jedynie w~przypadku języków inicjalnych oraz przedstawiam analogiczną analizę dla języków finalnych.

%poprawić po otrzymaniu wyników
