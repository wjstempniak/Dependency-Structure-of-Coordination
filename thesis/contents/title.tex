\maketitle

    \tytulang{Dependency structure of coordination -- an analysis of Universal Dependencies corpora}

    \begin{abstract}
        {Istnieje wiele poglądów na temat struktury składniowej koordynacji, czyli konstrukcji współrzędnie złożonych. Poprzednie badania \citep{przepiorkowski2023conjunct} pokazują metodę pozwalającą na testowanie ich poprawności. Wykorzystują one zasadę Dependency Length Minimization (DLM, \citealt{temperley2007minimization}), czyli tendencję do formułowania zdań tak, aby łączna długość relacji między słowami w~zdaniu była jak najmniejsza. 

%Niniejsza praca przedstawia analizę koordynacji pochodzących z~korpusów językowych opisanych w~standardzie Universal Dependencies \citep{de2021universal}. W~badaniu uwzględniono korpusy 13 języków. Zgodnie z~regułą DLM w~językach inicjalnych występuje tendencja do tworzenia konstrukcji współrzędnie złożonych z~dłuższym pierwszym członem, zaś w~językach finalnych z~dłuższym ostatnim członem. Badanie potwierdza tę prawidłowość. W~pracy analizowana jest także zależność między pozycją nadrzędnika koordynacji w~zdaniu oraz różnicą długości członów koordynacji.
%\emph{Badanie pokazuje ogólną tendencję do umieszczania krótszego członu bliżej nadrzędnika, co potwierdza przewidywania zasady DLM.}
}
    \end{abstract}

    \thispagestyle{empty}
    \setcounter{page}{3}

\chapter*{Podziękowania}

Pragnę złożyć serdeczne podziękowania mojemu promotorowi prof. dr. hab. Adamowi Przepiórkowskiemu za inspirację i~pomoc w~prowadzeniu badań oraz cierpliwość, wyrozumiałość i~zaangażowanie podczas kierowania moją pracą.

Ukończenie niniejszej pracy nie~byłoby możliwe bez pomocy mgr. Berkego \c{S}en\c{s}ekerci, któremu serdecznie dziękuję za pomoc w~dostosowaniu stosowanych przeze mnie algorytmów do~warunków języka tureckiego.

Dziękuję również Magdalenie Borysiak, Katarzynie Zrobek i Oskarowi Pruszyńskiemu za~koleżeńską pomoc w~prowadzeniu badań i zgłębianiu wiedzy na temat badanych przeze mnie zjawisk.
    
    \tableofcontents