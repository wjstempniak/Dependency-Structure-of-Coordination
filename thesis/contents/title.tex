\maketitle

    \tytulang{Dependency structure of coordination -- an analysis of Universal Dependencies corpora}

    \begin{abstract}{
Istnieje wiele poglądów na temat struktury zależnościowej koordynacji, czyli konstrukcji współrzędnie złożonej. W~literaturze opisane są cztery główne podejścia -- model praski, londyński, stanfordzki i~moskiewski \citep{popel2013coordination}.

Poprzednie badania \citep{przepiorkowski2023conjunct} opisują metodę pozwalającą na testowanie poprawności tych modeli. Polega ona na analizie tendencji do umieszczania krótszego członu koordynacji na początku konstrukcji współrzędnie złożonej. Wykorzystuje ona zasadę Dependency Length Minimization (DLM, \citealt{temperley2007minimization}), czyli tendencję do formułowania zdań tak, aby łączna długość relacji między słowami w~zdaniu była jak najmniejsza. \cite{przepiorkowski2023conjunct} na podstawie analizy koordynacji w korpusie języka angielskiego argumentują za poprawnością modeli symetrycznych, czyli podejścia praskiego i~londyńskiego.

Cztery główne modele struktury zależnościowej koordynacji zostały opracowane na podstawie analiz języków inicjalnych, czyli takich, w których głowy znajdują się zwykle na początku fraz. Jednak podejścia te mogą nie opisywać prawidłowo koordynacji w językach finalnych, tj. takich, w których głowa zwykle jest na końcu frazy. \cite{kanayama2018coordinate} postulują wprowadzenie alternatywnych modeli struktury zależnościowej koordynacji dla języków finalnych. W~niniejszej pracy przedstawiam przewidywania 12 modeli struktury zależnościowej koordynacji. Zestawiam je z~wynikami analizy korpusów językowych opisanych w standardzie Universal Dependencies \citep{de2021universal}. W~badaniu uwzględniono korpusy 13 języków, w tym 9 inicjalnych i~2 finalnych.

Wyniki badania potwierdzają występujące w języku angielskim tendencje zaobserwowane w pracy \cite{przepiorkowski2023conjunct}. Podobne zależności zaobserwowane są także w języku czeskim oraz w~łacinie. Niemniej jednak przewidywane tendencje nie występują w~pozostałych badanych językach. Pokazuję, że może to wynikać z~niewystarczającej ilości oraz złej jakości danych użytych w~badaniu. Proponuję poprawę metodologii i~dalsze badania dotyczące struktury zależnościowej koordynacji.
}
    \end{abstract}

    \thispagestyle{empty}
    \setcounter{page}{3}

\chapter*{Podziękowania}

Pragnę złożyć serdeczne podziękowania mojemu promotorowi prof. dr. hab. Adamowi Przepiórkowskiemu za inspirację i~pomoc w~prowadzeniu badań oraz cierpliwość, wyrozumiałość i~zaangażowanie podczas kierowania moją pracą.

Ukończenie niniejszej pracy nie~byłoby możliwe bez pomocy mgr. Berkego \c{S}en\c{s}ekerci, któremu serdecznie dziękuję za pomoc w~dostosowaniu stosowanych przeze mnie algorytmów do~warunków języka tureckiego.

Dziękuję również Magdalenie Borysiak, Katarzynie Zrobek i~Oskarowi Pruszyńskiemu za~koleżeńską pomoc w~prowadzeniu badań i~zgłębianiu wiedzy na temat badanych przeze mnie zjawisk.
    
    \tableofcontents