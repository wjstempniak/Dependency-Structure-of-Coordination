\chapter{Dyskusja wyników} \label{ch6}

\section{Replikacja poprzednich badań}

\subsection{Język angielski}

\cite{przepiorkowski2023conjunct} opisali dwie zależności dotyczące zmian tendencji do umieszczania krótszego członu koordynacji na jej początku wraz ze wzrostem różnic długości między członami koordynacji w~języku angielskim: tendencje pozytywną w koordynacjach z nadrzędnikiem po lewej stronie (L) oraz bez nadrzędnika (0) oraz (nieistotną statystycznie) tendencję negatywną w~koordynacjach z nadrzędnikiem po prawej stronie (R).

Niniejsza analiza replikuje obserwację pozytywną dotyczącą koordynacji (L) oraz~(0). W przypadku koordynacji typu (R) zależność jest pozytywna, gdy długość liczona jest w~znakach ($p<0,001$) i~sylabach ($p=0,003$).
Gdy długość członów liczona jest w słowach, tendencja jest negatywna i nieistotna statystycznie ($p=0,755$).

Te wyniki należy interpretować w kontekście efektu DLM. Polega on na minimalizacji łącznej długości relacji zależnościowych w~języku. Długość zależności może być liczona na różne sposoby. Najlepszym z nich jest najprawdopodobniej złożoność syntaktyczna frazy \citep{lohmann2014english}. Spośród stosowanych przez mnie miar (słowa, sylaby, znaki) najbliższą złożoności syntaktycznej są słowa. W~związku z~tym uznaję wyniki dotyczące długości członów w~słowach za istotniejsze od pozostałych.

Oznacza to, że~niniejsza analiza replikuje badanie \cite{przepiorkowski2023conjunct} w~zakresie opisu relacji między różnicą długości członów koordynacji a tendencją do umieszczania krótszego członu jako pierwszego w~języku angielskim.

\subsection{Języki słowiańskie}

W języku czeskim zależność dotycząca koordynacji (R) jest podobna do tej w~języku angielskim. Jest to tendencja spadkowa istotna statystycznie na poziomie $p=0,019$. Tendencje w~języku polskim oraz rosyjskim nie są istotne statystycznie. Może to oznaczać, że w~językach słowiańskich występuje ta sama zależność między pozycją nadrzędnika a zmianą odsetka koordynacji (R-L) względem (R), co w~języku angielskim. Znaczy to, że niniejsze badanie rozszerza wyniki pracy \cite{przepiorkowski2023conjunct} na język czeski oraz nie wyklucza, że omawiane tendencje występują w~pozostałych językach słowiańskich.

\subsection{Języki romańskie}

Wyniki dla analizowanych języków romańskich pokazują jednoznaczną istotną statystycznie tendencję pozytywną we wszystkich rozpatrywanych przypadkach. Oznacza to, że w~językach włoskim, hiszpańskim i~portugalskim krótszy człon koordynacji znajduje się tym częściej jako pierwszy, im krótszy jest on od ostatniego członu niezależnie od pozycji nadrzędnika. 

W przypadku języka rumuńskiego omawiana tendencja nie jest istotna statystycznie ($p=0,29$). Warto zauważyć, że język ten pomimo należenia do grupy języków romańskich jest pod silnym wpływem języków słowiańskich.

\subsection{Języki mieszane}

W przypadku języka niemieckiego widoczne są podobne tendencje, jak w~przypadku języków romańskich, zaś w~łacinie występuje taka zależność, jak w~języku angielskim i~w~językach słowiańskich. Ponieważ metodologia niniejszej pracy nie zakładała przewidywań dotyczących tendencji występujących w~tych językach, nie interpretuję tych wyników w~kontekście replikacji badania \cite{przepiorkowski2023conjunct}.

\subsection{Języki finalne}

W języku koreańskim widoczna jest ewidentna tendencja wzrostowa we~wszystkich przypadkach. Jest ona znacznie istotna statystycznie ($p<0,001$) w przypadku koordynacji pozbawionych nadrzędnika, istotna na poziomie $p=0,002$ dla koordynacji (L) oraz na poziomie $p=0,04$ dla koordynacji (R). Jest to podobna tendencja do tej zaobserwowanej w~językach romańskich i~języku niemieckim. W przypadku języka tureckiego tendencja jest wyraźnie słabsza. W przypadku koordynacji (L) nie jest ona istotna statystycznie ($p=0,29$).

\section{Przewidywania modeli struktury zależnościowej koordynacji}

\subsection{Języki inicjalne}

Jak zostało to omówione w punkcie \ref{podejścia}, istnieją cztery przyjmowane podejścia do struktury zależnościowej koordynacji:

\begin{table}[H]

\centering

\begin{tabular}{c c}

\textbf{Spójnikowe/Praskie}

&

\textbf{Wielogłowe/Londyńskie}

\\

\begin{dependency}[hide label, edge unit distance=0.5ex]

        \begin{deptext}
        $\odot$\&$\square$\&$\square$\&$\square$\&,\&$\square$\&$\square$\&$\square$\&$\boxdot$\&$\square$\&$\square$\&$\square$\\
            \end{deptext}
            \depedge{1}{9}{}
            \depedge{9}{2}{}
            \depedge{9}{6}{}
            \depedge{9}{10}{}
            \wordgroup{1}{2}{4}{c1}
            \wordgroup{1}{6}{8}{c2}
            \wordgroup{1}{10}{12}{c3}
        \end{dependency}

&

\begin{dependency}[hide label, edge unit distance=0.5ex]

        \begin{deptext}
        $\odot$\&$\square$\&$\square$\&$\square$\&,\&$\square$\&$\square$\&$\square$\&$\boxdot$\&$\square$\&$\square$\&$\square$\\
            \end{deptext}
            \depedge{1}{2}{}
            \depedge{1}{6}{}
            \depedge{1}{10}{}
            \depedge{10}{9}{}
            \wordgroup{1}{2}{4}{c1}
            \wordgroup{1}{6}{8}{c2}
            \wordgroup{1}{10}{12}{c3}
        \end{dependency}

\vspace{.5cm}
\\ 

\textbf{Bukietowe/Stanfordzkie}

&

\textbf{Łańcuchowe/Moskiewskie}

\\

\begin{dependency}[hide label, edge unit distance=0.5ex]

        \begin{deptext}
        $\odot$\&$\square$\&$\square$\&$\square$\&,\&$\square$\&$\square$\&$\square$\&$\boxdot$\&$\square$\&$\square$\&$\square$\\
            \end{deptext}
            \depedge{1}{2}{}
            \depedge{2}{6}{}
            \depedge{2}{10}{}
            \depedge{10}{9}{}
            \wordgroup{1}{2}{4}{c1}
            \wordgroup{1}{6}{8}{c2}
            \wordgroup{1}{10}{12}{c3}
        \end{dependency}

&

\begin{dependency}[hide label, edge unit distance=0.5ex]

        \begin{deptext}
        $\odot$\&$\square$\&$\square$\&$\square$\&,\&$\square$\&$\square$\&$\square$\&$\boxdot$\&$\square$\&$\square$\&$\square$\\
            \end{deptext}
            \depedge{1}{2}{}
            \depedge{2}{6}{}
            \depedge{6}{9}{}
            \depedge{9}{10}{}
            \wordgroup{1}{2}{4}{c1}
            \wordgroup{1}{6}{8}{c2}
            \wordgroup{1}{10}{12}{c3}
        \end{dependency}

\end{tabular}
\end{table}


Podejścia praskie i~londyńskie nazywane są symetrycznymi, zaś stanfordzkie i~moskiewskie -- asymetrycznymi. \cite{przepiorkowski2023conjunct} na podstawie przeprowadzonej analizy korpusowej języka angielskiego argumentują, że jedynie podejścia symetryczne (praskie i~londyńskie) mogą poprawnie opisywać strukturę zależnościową konstrukcji współrzędnie złożonej. Opierają swoje rozumowanie na predykcjach modeli dotyczących zmiany tendencji do umieszczania krótszego członu koordynacji na jej początku wraz ze~wzrostem różnicy długości członów.

Wszystkie cztery podejścia, zgodnie z faktami, przewidują tendencję pozytywną w~przypadku koordynacji (L). Dla koordynacji (0) podejścia praskie, stanfordzkie i~moskiewskie poprawnie przewidują tendencje pozytywne, zaś podejście londyńskie (wbrew faktom) przewiduje brak zależności. Kluczowa jest zależność dotycząca koordynacji (R). Podejścia symetryczne przewidują tendencję spadkową, zaś podejścia asymetryczne tendencję wzrostową.

Ponieważ w~języku angielskim omawiana tendencja dla koordynacji (R) jest negatywna, \cite{przepiorkowski2023conjunct} odrzucają podejścia asymetryczne. Bronią jednocześnie podejścia londyńskiego na podstawie argumentu o gramatykalizacji. Zgodnie z nim fakt, że koordynacje (L) występują w~języku angielskim znacznie częściej niż (R) powoduje, że umieszczanie pierwszego członu jako pierwszego mogło stać się regułą.

Wyniki przeprowadzonej przeze mnie analizy w~zakresie języka angielskiego i~języków słowiańskich nie zaprzeczają przywołanej wyżej argumentacji. Niemniej jednak wyniki dotyczące pozostałych języków inicjalnych wskazują na występowanie zupełnie innych tendencji w~językach romańskich. Istnieją przynajmniej dwa możliwe wyjaśnienia takiego zjawiska.

Zależności widoczne w~językach romańskich są zgodne z przewidywaniami modeli asymetrycznych. Istnieje więc możliwość, że języki romańskie posiadają odrębną strukturę zależnościową koordynacji niż ta występująca w~języku angielskim i~w~językach słowiańskich.

Drugie wyjaśnienie takiego stanu rzeczy jest oparte na argumencie o gramatykalizacji. Możliwe że gramatykalizacja umieszczania krótszego członu po lewej stronie koordynacji jest w~przypadku języków romańskich tak silna, że niezależnie od faktycznej struktury zależnościowej koordynacji efekt DLM nie ma większego wpływu na ustawienie członów. Innymi słowy, krótszy człon zawsze występuje częściej z lewej strony tym częściej, im większa jest różnica długości członów niezależnie od pozycji nadrzędnika.

\subsection{Języki finalne}

Tendencje zaobserwowane w~językach finalnych są znacznie trudniejsze do wyjaśnienia na podstawie efektu DLM. Jak zostało to pokazane w punkcie \ref{wszystkie-podejścia}, istnieje 16 możliwych podejść do struktury zależnościowej koordynacji:

\begin{table}[h]
\centering
\begin{tabular}{ c c c c }
(A)
\begin{dependency}[theme = simple, edge unit distance=0.5ex, baseline=-\the\dimexpr\fontdimen22\textfont2\relax]
        \begin{deptext}
        $\odot$ \& $\square$ \& $\boxdot$ \& $\square$\\
            \end{deptext}
		\depedge{1}{3}{}
		\depedge{3}{2}{}
		\depedge{3}{4}{}
        \end{dependency}
&
(B)
\begin{dependency}[theme = simple, edge unit distance=0.5ex, baseline=-\the\dimexpr\fontdimen22\textfont2\relax]
        \begin{deptext}
         $\odot$ \& $\square$ \& $\boxdot$ \& $\square$\\
	\end{deptext}
		\depedge{1}{2}{}
		\depedge{1}{4}{}
		\depedge{4}{3}{}
        \end{dependency}
& 
(C)
\begin{dependency}[theme = simple, edge unit distance=0.5ex, baseline=-\the\dimexpr\fontdimen22\textfont2\relax]
        \begin{deptext}
        $\odot$ \& $\square$ \& $\boxdot$ \& $\square$\\
            \end{deptext}
		\depedge{1}{2}{}
		\depedge{1}{4}{}
		\depedge{2}{3}{}
        \end{dependency}
& 
(D)
\begin{dependency}[theme = simple, edge unit distance=0.5ex, baseline=-\the\dimexpr\fontdimen22\textfont2\relax]
        \begin{deptext}
        $\odot$ \& $\square$ \& $\boxdot$ \& $\square$\\
            \end{deptext}
		\depedge{1}{2}{}
		\depedge{2}{3}{}
		\depedge{3}{4}{}
        \end{dependency}
\\ 
(E)
\begin{dependency}[theme = simple, edge unit distance=0.5ex, baseline=-\the\dimexpr\fontdimen22\textfont2\relax]
        \begin{deptext}
        $\odot$ \& $\square$ \& $\boxdot$ \& $\square$\\
            \end{deptext}
		\depedge{1}{4}{}
		\depedge{4}{3}{}
		\depedge{3}{2}{}
        \end{dependency}
&
(F)
\begin{dependency}[theme = simple, edge unit distance=0.5ex, baseline=-\the\dimexpr\fontdimen22\textfont2\relax]
        \begin{deptext}
        $\odot$ \& $\square$ \& $\boxdot$ \& $\square$\\
            \end{deptext}
		\depedge{1}{2}{}
		\depedge{2}{4}{}
		\depedge{4}{3}{}
        \end{dependency}
& 
(G)
\begin{dependency}[theme = simple, edge unit distance=0.5ex, baseline=-\the\dimexpr\fontdimen22\textfont2\relax]
        \begin{deptext}
        $\odot$ \& $\square$ \& $\boxdot$ \& $\square$\\
            \end{deptext}
		\depedge{1}{4}{}
		\depedge{4}{2}{}
		\depedge{4}{3}{}
        \end{dependency}
& 
(H)
\begin{dependency}[theme = simple, edge unit distance=0.5ex, baseline=-\the\dimexpr\fontdimen22\textfont2\relax]
        \begin{deptext}
        $\odot$ \& $\square$ \& $\boxdot$ \& $\square$\\
            \end{deptext}
		\depedge{1}{2}{}
		\depedge{2}{4}{}
		\depedge{2}{3}{}
        \end{dependency}
\\  
(I)
\begin{dependency}[theme = simple, edge unit distance=0.5ex, baseline=-\the\dimexpr\fontdimen22\textfont2\relax]
        \begin{deptext}
        $\odot$ \& $\square$ \& $\boxdot$ \& $\square$\\
            \end{deptext}
		\depedge{1}{4}{}
		\depedge{4}{2}{}
		\depedge{2}{3}{}
        \end{dependency}
&
(J)
\begin{dependency}[theme = simple, edge unit distance=0.5ex, baseline=-\the\dimexpr\fontdimen22\textfont2\relax]
        \begin{deptext}
        $\odot$ \& $\square$ \& $\boxdot$ \& $\square$\\
            \end{deptext}
		\depedge{1}{2}{}
		\depedge{1}{3}{}
		\depedge{1}{4}{}
        \end{dependency}
& 
(K)
\begin{dependency}[theme = simple, edge unit distance=0.5ex, baseline=-\the\dimexpr\fontdimen22\textfont2\relax]
        \begin{deptext}
        $\odot$ \& $\square$ \& $\boxdot$ \& $\square$\\
            \end{deptext}
		\depedge{1}{2}{}
		\depedge{1}{3}{}
		\depedge{3}{4}{}
        \end{dependency}
& 
(L)
\begin{dependency}[theme = simple, edge unit distance=0.5ex, baseline=-\the\dimexpr\fontdimen22\textfont2\relax]
        \begin{deptext}
        $\odot$ \& $\square$ \& $\boxdot$ \& $\square$\\
            \end{deptext}
		\depedge{1}{3}{}
		\depedge{1}{4}{}
		\depedge{3}{2}{}
        \end{dependency}
 \\
(M)
\begin{dependency}[theme = simple,  edge unit distance=0.5ex, baseline=-\the\dimexpr\fontdimen22\textfont2\relax]
        \begin{deptext}
        $\odot$ \& $\square$ \& $\boxdot$ \& $\square$\\
            \end{deptext}
		\depedge{1}{3}{}
		\depedge{1}{4}{}
		\depedge{4}{2}{}
        \end{dependency}
&
(N)
\begin{dependency}[theme = simple, edge unit distance=0.5ex, baseline=-\the\dimexpr\fontdimen22\textfont2\relax]
        \begin{deptext}
        $\odot$ \& $\square$ \& $\boxdot$ \& $\square$\\
            \end{deptext}
		\depedge{1}{3}{}
		\depedge{1}{2}{}
		\depedge{2}{4}{}
        \end{dependency}
& 
(O)
\begin{dependency}[theme = simple, edge unit distance=0.5ex, baseline=-\the\dimexpr\fontdimen22\textfont2\relax]
        \begin{deptext}
        $\odot$ \& $\square$ \& $\boxdot$ \& $\square$\\
            \end{deptext}
		\depedge{1}{3}{}
		\depedge{3}{2}{}
		\depedge{2}{4}{}
        \end{dependency}
& 
(P)
\begin{dependency}[theme = simple, edge unit distance=0.5ex, baseline=-\the\dimexpr\fontdimen22\textfont2\relax]
        \begin{deptext}
        $\odot$ \& $\square$ \& $\boxdot$ \& $\square$\\
            \end{deptext}
		\depedge{1}{3}{}
		\depedge{3}{4}{}
		\depedge{4}{2}{}
        \end{dependency}
\end{tabular}
\end{table}


Jak pokazuję w~rozdziale \ref{ch3}, żadne z nich nie przewiduje zaobserwowanej tendencji pozytywnej dla koordynacji (R). Żadne z podejść nie przewiduje też bardziej pozytywnej tendencji dla koordynacji (L) niż dla koordynacji (R). Oznacza to, że nie istnieje struktura zależnościowa koordynacji, która może tłumaczyć uzyskane wyniki. Jednocześnie niemożliwe jest, żeby koordynacja nie posiadała żadnej struktury zależnościowej. 

Uzyskanie niewyjaśnialnych wyników może oznaczać, że~metodologia niniejszej pracy jest niewystarczająca w~celu poprawnego i~jednoznacznego opisu struktury zależnościowej koordynacji.

\section{Wyjaśnienia wyników}

\subsection{Gramatykalizacja krótszego prawego członu} \label{gramatykalizacja}

Z~Tabel \ref{tab:czeski+angielski} i~\ref{tab:niemiecki+koreański} wynika, że we wszystkich badanych językach, niezależnie od pozycji nadrzędnika krótszy człon koordynacji znajduje się częściej na jej początku. Może to oznaczać, że umieszczanie najkrótszego jako pierwszego uległo gramatykalizacji.

W przypadku języków inicjalnych, przyczyny tej gramatykalizacji są proste w~wyjaśnieniu. Każde z~czterech omawianych podejść zakłada, że w~przypadku  koordynacji z~nadrzędnikiem po lewej stronie wzrost różnicy długości członów przekłada się na częstsze umieszczanie krótszego członu na początku koordynacji. Ponieważ w~językach inicjalnych koordynacje z~nadrzędnikiem po lewej stronie są znacznie częstsze od pozostałych (a w~wielu przypadkach stanowią ponad połowę konstrukcji współrzędnie złożonych), umieszczanie krótszego członu po lewej stronie stało się w~tych językach regułą. 

Niemniej jednak zastosowanie analogicznego rozumowania nie wyjaśnia tendencji uzyskanych w językach finalnych. Żadne z~analizowanych podejść nie zakłada częstszego umieszczania krótszego członu na początku koordynacji wraz ze wzrostem różnicy długości członów niezależnie od pozycji nadrzędnika. Oznacza to, że umieszczanie krótszego członu po lewej stronie nie może wynikać z tego, że koordynacje (R) są częstsze w~językach finalnych.

Jeśli w~językach finalnych umieszczanie krótszego członu po lewej stronie koordynacji uległo gramatykalizacji, musi wynikać to z~innych przyczyn, niż ze~struktury składniowej koordynacji. 

\subsection{Monotoniczność tendencji}

Dependency Length Minimisation jest zjawiskiem polegającym na możliwym skracaniu relacji zależnościowych występujących w~zdaniach. Jego wpływ na ustawianie kolejności członów koordynacji można traktować jako funkcję bezwzględnej różnicy długości członów w~prawdopodobieństwo wystąpienia krótszego członu po lewej stronie. Nie wiadomo nic na temat działania tej funkcji, oprócz tego, że jest ona monotoniczna.

Niemniej jednak istnieje możliwość, że pomimo, że efekt DLM ma charakter funkcji monotonicznej, tendencja do zmiany proporcji umieszczania krótszego członu po lewej stronie koordynacji wraz ze zmianą różnicy długości członów nie jest monotoniczna. 

\cite{przepiorkowski2024argument} przeprowadzili replikację analizy z~pracy \cite{przepiorkowski2023conjunct} na większym korpusie języka angielskiego. Przeanalizowali 11 502 053 koordynacji. Dla porównania, \cite{przepiorkowski2023conjunct} zbadali 21 825 konstrukcji współrzędnie złożonych, a w~niniejszej pracy przeanalizowano 21 013 koordynacji języka angielskiego. Wyniki dotyczące koordynacji (L) i~(0) potwierdziły tendencje zaobserwowane z pracy \cite{przepiorkowski2023conjunct}. 

W przypadku koordynacji (R) \cite{przepiorkowski2024argument} zaobserwowali niemonotoniczną zależność między zmianą odsetka występowania krótszych członów na początku koordynacji a różnicą długości członów. Dla niewielkiej różnicy długości (nie przekraczającej 4 słów lub 30 znaków) dla koordynacji (R) zaobserwowano tendencję rosnącą, zaś dla większych różnic tendencję malejącą. 

Możliwe, że w~przypadku badanych przeze mnie języków istnieją podobne zależności. Jeśli tak jest, taki kształt relacji  może wynikać z interakcji między dwiema przeciwnymi tendencjami. W koordynacjach (R) efekt DLM składa się na tendencję spadkową, zaś inne przyczyny na tendencje wzrostową. W przypadku konstrukcji z małą różnicą długości członów efekt DLM jest słaby i~nie jest widoczny. Dopiero~gdy różnica długości przekracza pewien próg (na przykład czterech słów) efekt DLM staje się silniejszy niż pozostałe tendencje wzrostowe.

Niestety, weryfikacja powyższych spekulacji za pomocą korpusów zależnościowych UD jest niemożliwa. W~analizowanych danych znajduje się zbyt mało koordynacji (R) o dużej różnicy długości członów, żeby można było na ich podstawie utworzyć istotny statystycznie model.

\subsection{Przyczyny tendencji wzrostowych}

\cite{przepiorkowski2023conjunct} twierdzą, że tendencje wzrostowe dla koordynacji (0) mogą wynikać z~gramatykalizacji umieszczania krótszego członu po lewej stronie. Jak pokazuję w~puncie \ref{gramatykalizacja}, wyjaśnienie to może opisywać jedynie wyniki uzyskane dla języków inicjalnych. Gramatykalizacja ta nie może zachodzić w~językach inicjalnych lub może zachodzić w~inny sposób, niż twierdzą \cite{przepiorkowski2023conjunct}.

W niniejszym punkcie przedstawiam argument mogący stanowić wyjaśnienie występowania różnych tendencji dotyczących koordynacji (R) w~badanych przeze mnie językach.

W rozdziale \ref{ch2} opisuję różne przyczyny, dla których w~języku występuje naturalna tendencja do umieszczania krótszego członu koordynacji po lewej stronie. Należą do nich czynniki pragmatyczne, psycholingwistyczne (argument o częstym używaniu krótszych słów) oraz  wynikające z prozodii (argument o rozłożeniu sylab akcentowanych). 

\cite{wright2005ladies} wskazują, że czynniki wynikające z prozodii mają szczególnie mocny wpływ na koordynacje krótkie, tj. o członach jedno- i~dwusylabowych. W przypadku takich koordynacji różnica długości członów jest zwykle niewielka, więc efekt DLM może być pomijalny.

Jak wskazują \cite{mcdonald1993word}, koordynacje o krótkich członach są konstruowane tak, żeby akcentowane sylaby tworzyły rytm. Pokazują to poniższe przykłady:

\begin{exe}
\ex \label{jan+maria-obiad} {
\metrics{   _   u |   _   u   |  _   u | _  u  }
        {[[Jan] i | [Ma-ria]] |  \textit{zje}-\textit{dli} | o-biad.}
        }
        
\ex \label{maria+jan-obiad} {
\metrics{   _   u  u   _      _   u _   u  }
        {[[Ma-ria] i [Jan]] \textit{zje}-\textit{dli} o-biad.}
        }
        
\ex \label{obiad-jan+maria} {
\metrics{_   u  |   _   u |    _   u |   _   u   }
        {O-biad | \textit{zje}-\textit{dli} | [[Jan] i | [Ma-ria]].}
        }

\ex \label{obiad-maria+jan} {
\metrics{_   u    _   u    _   u  u   _    }
        {O-biad \textit{zje}-\textit{dli} [[Ma-ria] i [Jan]].}
        }
\end{exe}
        
W przykładach \eqref{jan+maria-obiad} i~\eqref{maria+jan-obiad} nadrzędnik koordynacji \textit{zjedli} znajduje się po lewej stronie, zaś w~przykładach \eqref{obiad-jan+maria} i~\eqref{obiad-maria+jan} po prawej stronie. Rytm oparty o stopy metryczne powstaje tylko w~zdaniach, w~których krótszy człon \textit{Jan} występuje jako pierwszy, tj. w~\eqref{jan+maria-obiad} i~\eqref{obiad-jan+maria}. W tych przykładach rozkład sylab akcentowanych jest niezależny od pozycji nadrzędnika. Zgodnie z argumentacją przedstawioną w~pracy \cite{mcdonald1993word} to właśnie zdania \eqref{jan+maria-obiad} i~\eqref{obiad-jan+maria} mają większą szansę na pojawienie się w~języku naturalnym niż zdania \eqref{maria+jan-obiad} i~\eqref{obiad-maria+jan}\footnote{
Warto zauważyć, że przykład zdań \eqref{jan+maria-obiad}--\eqref{obiad-maria+jan} stanowi szczególny przypadek, w którym jeden z członów koordynacji posiada jedną, zaś drugi dwie sylaby. W przypadku koordynacji, w której człony miałyby odpowiednio dwie i trzy sylaby, tendencje byłyby odwrotne. Podane przykłady nie mają na celu wykazania, że prozodia ma większy wpływ na ustawienie członów koordynacji niż efekt DLM ani udowodnienia częstości występowania koordynacji z członami jedno- i dwusylabowymi. Zdania \eqref{jan+maria-obiad}--\eqref{obiad-maria+jan} mają na celu jedynie pokazać, że mogą istnieć sytuacje, w których rozkład sylab akcentowanych ma większy wpływ na ustawienie członów koordynacji niż efekt DLM. Do określenia, jak częste jest to zjawisko i jaka jest jego faktyczna interakcja z efektem DLM, należałoby przeprowadzić osobne badanie.
}.

To samo rozumowanie nie ma zastosowania w~przypadku koordynacji z większą różnicą długości członów.

  \begin{exe}
\ex \label{jan+maria-obiad-długie} {
\metrics{   u  _  u   _  u   _   u   u  _   u     _   u    _  u u   _   u   _   u    _    u _   u  }
        {[[Wy-so-ki bru-net Jan] i [ko-bie-ta w~śre-dnim wie-ku i-mie-niem Ma-ria]] \textit{zje}-\textit{dli} o-biad.}
        }
        
\ex \label{maria+jan-obiad-długie} {
\metrics{   u  _   u     _   u    _  u u   _   u   _  u   u   u  _  u   _  u   _      _   u  _  u  }
        {[[Ko-bie-ta w~śre-dnim wie-ku i-mie-niem Ma-ria] i [wy-so-ki bru-net Jan]] \textit{zje}-\textit{dli} o-biad.}
        }
        
\ex \label{obiad-jan+maria-długie} {
\metrics{_   u    _   u    u  _  u   _  u   _   u    u  _   u     _   u   _  u u   _   u  _   u    }
        {O-biad \textit{zje}-\textit{dli} [[wy-so-ki bru-net Jan] i [ko-bie-ta w~śre-dnim wie-ku i-mie-niem Ma-ria]].}
        }

\ex \label{obiad-maria+jan-długie} {
\metrics{_   u    _   u    u  _  u   _  u   _   u   u  _   u     _   u    _  u _   u   _  u   u    }
        {O-biad \textit{zje}-\textit{dli} [[ko-bie-ta w~śre-dnim wie-ku i-mie-niem Ma-ria] i [wy-so-ki bru-net Jan]].}
        }
\end{exe}      

Dodatkowe podrzędniki głów członów \textit{Jan} oraz \textit{Maria} w~przykładach \eqref{jan+maria-obiad-długie}--\eqref{obiad-maria+jan-długie} zaburzają rytm we wszystkich czterech przypadkach. Niezależnie od pozycji krótszego członu osiągnięcie pożądanej sekwencji naprzemiennie występujących sylab krótkich i~długich jest niemożliwy\footnote{\setstretch{1.5}
Rytm zawsze występuje w~części zdania -- np. fraza \metrics{u _ | u _ | u _}{Wy-so | ki bru | net Jan} dzieli się na trzy stopy jambiczne. Kluczowe jest ustawienie sylab na krawędziach członu: we~fragmencie \metrics{_ u u u _ u}{Ma-ria i wy-so-ki} nie występuje rytm.}.
Powoduje to, że omawiany efekt prozodyczny nie ma w~tych przykładach zastosowania. W takiej sytuacji większe znaczenie dla ustawienia kolejności członów może mieć efekt DLM. Jeśli prawidłowe są podejścia symetryczne (jak argumentują \citealt{przepiorkowski2023conjunct}), to zdanie \eqref{maria+jan-obiad-długie} ma większą szansę pojawić się w~języku, niż zdanie \eqref{jan+maria-obiad-długie}, zaś zdanie \eqref{obiad-jan+maria-długie} ma większą szansę niż \eqref{obiad-maria+jan-długie}.

Kluczową różnicą między zdaniami \eqref{jan+maria-obiad}--\eqref{obiad-maria+jan} i~\eqref{jan+maria-obiad-długie}--\eqref{obiad-maria+jan-długie} jest długość członów występujących w~nich koordynacji. Możliwe, że w~przypadku krótszych członów rytm występuje częściej niż w~przypadku dłuższych, ponieważ każdy dodatkowy podrzędnik głowy członu ma szansę na jego zaburzenie. W takiej sytuacji w~koordynacjach z krótszymi członami większy wpływ ma efekt ustawienia sylab akcentowanych, zaś w~przypadku zdań z dłuższymi koordynacjami efekt DLM.

Koordynacje z dużą różnicą długości członów (przekraczającą kilka słów) są możliwe tylko wtedy, kiedy jeden z członów jest długi. Możliwe,  że~w~przypadku koordynacji z większą różnicą długości członów jest więcej koordynacji z długimi członami, niż w~przypadku koordynacji z małą różnicą długości członów. Jeśli to prawda, zgodnie z omówioną powyżej hipotezą efekt DLM ma większy wpływ od efektu wynikającego z prozodii tylko w~przypadku koordynacji z różnicą długości członów przekraczającą kilka słów. W takim przypadku interakcja tych dwóch efektów jest jednym z~możliwych wyjaśnień wyników uzyskanych w~badaniu \citep{przepiorkowski2024argument}

To zjawisko może również tłumaczyć zaobserwowane w~niniejszej pracy różnice między językami. Typowe rozkłady sylab akcentowanych różnią się w~zależności od języka. Ponadto, mogą być one zależne w~większym stopniu od przynależności do danej rodziny językowej niż związane z inicjalnością lub finalnością języka. Jeśli w~językach romańskich i~języku koreańskim rytm zdania ma większy wpływ na ułożenie kolejności słów w~zdaniu i~nie jest tak łatwo zaburzany jak w~języku angielskim i~językach słowiańskich, może to być wyjaśnieniem otrzymanych wyników.

Weryfikacja postawionej w~niniejszym punkcie hipotezy nie jest możliwa na podstawie metod i~danych użytych w~niniejszej pracy. W~następnym rozdziale~\ref{ch7} omawiam przyczyny tego faktu oraz propozycje badań mogących sprawdzić opisaną przeze mnie hipotezę.
