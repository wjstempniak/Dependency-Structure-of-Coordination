\chapter{Rozszerzone zależności składniowe}

\section{Enhanced Dependencies}

W niniejszej pracy analizuję relacje składniowe w~formie drzew zależnościowych. Oznacza to, że zakładam, że relacje składniowe w~zdaniach tworzą drzewa. Jest to dość powszechne w~językoznawstwie założenie, stojące u podstaw gramatyki zależnościowej. Jednak nie musi być ono prawdziwe. 

Jak zauważają \cite{de2021universal}, w~języku istnieje wiele konstrukcji, w~przypadku których opis relacji zależnościowych za pomocą drzew nie jest optymalny. Należą do nich m.in. koordynacje.

Z~tego powodu twórcy Universal Dependencies stworzyli rozszerzenie standardu o~nazwie Enhanced Dependencies (ED)\footnote{\url{https://universaldependencies.org/u/overview/enhanced-syntax.html}}. Przykład \eqref{UD} pokazuje drzewo zależnościowe zdania \eqref{tworzyć+konstruować} opisane według zwykłego UD. \eqref{ED} jest drzewem zdania \eqref{tworzyć+konstruować} opisanym według ED. Rozszerzona zależność \texttt{xcomp} została wyróżniona.

\begin{exe}
\ex \label{tworzyć+konstruować}
Chcą tworzyć, konstruować.
 
\ex \label{UD}
\begin{dependency}[baseline=-\the\dimexpr\fontdimen22\textfont2\relax]
\begin{deptext}[column sep=1em]
Chcą \& tworzyć \& , \& konstruować \& .  \\ 
\end{deptext}
\deproot{1}{root}
\depedge{1}{2}{xcomp}
\depedge{4}{3}{punct}
\depedge{2}{4}{conj}
\depedge{1}{5}{punct}
\end{dependency}

\ex \label{ED}
\begin{dependency}[baseline=-\the\dimexpr\fontdimen22\textfont2\relax]
\begin{deptext}[column sep=1em]
Chcą \& tworzyć \& , \& konstruować \& .  \\ 
\end{deptext}
\deproot{1}{root}
\depedge{1}{2}{xcomp}
\depedge{4}{3}{punct}
\depedge{2}{4}{conj}
\depedge{1}{5}{punct}
\depedge[edge style = {blue, thick}, label style = {thick, draw=blue, text=blue}]{1}{4}{\textbf{xcomp}}
\end{dependency}
\citep{przepiorkowski2018lexical}
\end{exe}

\cite{przepiorkowski2023conjunct} zauważają, że uwzględnienie rozszerzonych relacji zależnościowych w~strukturze koordynacji zmienia przewidywania modeli dotyczące omawianych tendencji. Prezentują model struktury zależnościowej koordynacji oparty na założeniach ED:

\begin{exe}
\ex \label{ED-model}
\begin{dependency}[theme = simple, edge unit distance=0.5ex, baseline=-\the\dimexpr\fontdimen22\textfont2\relax]
        \begin{deptext}
        $\odot$ \& $\square$ \& $\boxdot$ \& $\square$\\
            \end{deptext}
		\depedge{1}{2}{}
		\depedge{1}{4}{}
		\depedge{2}{4}{}
		\depedge{4}{3}{}
        \end{dependency}
\end{exe}

\section{Języki inicjalne}

W językach inicjalnych model przewiduje następujące długości relacji:

\begin{table}[H]
\begin{tabular}{lcllcl}

(L-L) &

\begin{dependency}[hide label, edge unit distance=0.5ex, baseline=-\the\dimexpr\fontdimen22\textfont2\relax]
        \begin{deptext}
        $\odot$\&$\square$\&a\&$\boxdot$\&$\square$\&a+b\\
        \end{deptext}
		\depedge{1}{2}{}
		\depedge{1}{5}{}
		\depedge{5}{4}{}
		\depedge{2}{5}{}
        \wordgroup{1}{2}{3}{L}
        \wordgroup{1}{5}{6}{R}
        \end{dependency}

& $S=2a$ & 

(L-R) &

\begin{dependency}[hide label, edge unit distance=0.5ex, baseline=-\the\dimexpr\fontdimen22\textfont2\relax]
        \begin{deptext}
        $\odot$\&$\square$\&a+b\&$\boxdot$\&$\square$\&a\\
        \end{deptext}
		\depedge{1}{2}{}
		\depedge{1}{5}{}
		\depedge{5}{4}{}
		\depedge{2}{5}{}
		\wordgroup{1}{2}{3}{L}
		\wordgroup{1}{5}{6}{R}
        \end{dependency}
        
& $S=2a+2b$ \\ 

(0-L) &

\begin{dependency}[hide label, edge unit distance=0.5ex, baseline=-\the\dimexpr\fontdimen22\textfont2\relax]
        \begin{deptext}
        $\square$\&a\&$\boxdot$\&$\square$\&a+b\\
        \end{deptext}
		\depedge{4}{3}{}
		\depedge{1}{4}{}
        \wordgroup{1}{1}{2}{L}
        \wordgroup{1}{4}{5}{R}
        \end{dependency}

& $S=a$ & 

(0-R) &

\begin{dependency}[hide label, edge unit distance=0.5ex, baseline=-\the\dimexpr\fontdimen22\textfont2\relax]
        \begin{deptext}
        $\square$\&a+b\&$\boxdot$\&$\square$\&a\\
        \end{deptext}
		\depedge{4}{3}{}
		\depedge{1}{4}{}
        \wordgroup{1}{1}{2}{L}
        \wordgroup{1}{4}{5}{R}
        \end{dependency}
        
& $S=a+b$ \\

(R-L) &

\begin{dependency}[hide label,edge unit distance=0.5ex, baseline=-\the\dimexpr\fontdimen22\textfont2\relax]
        \begin{deptext}
        $\square$\&a\&$\boxdot$\&$\square$\&a+b\&$\odot$\\
        \end{deptext}
		\depedge{6}{1}{}
		\depedge{6}{4}{}
		\depedge{4}{3}{}
		\depedge{4}{1}{}
		\wordgroup{1}{1}{2}{L}
		\wordgroup{1}{4}{5}{R}
        \end{dependency}
        
& $S=4a+2b$ &

(R-R) &

\begin{dependency}[hide label, edge unit distance=0.5ex, baseline=-\the\dimexpr\fontdimen22\textfont2\relax]
        \begin{deptext}
        $\square$\&a+b\&$\boxdot$\&$\square$\&a\&$\odot$\\
        \end{deptext}
		\depedge{6}{1}{}
		\depedge{6}{4}{}
		\depedge{4}{3}{}
		\depedge{4}{1}{}
        \wordgroup{1}{1}{2}{L}
        \wordgroup{1}{4}{5}{R}
        \end{dependency}

& $S=4a+2b$ \\

\end{tabular}
\end{table}

Wraz~ze wzrostem różnicy długości członów (\emph{b}):
\begin{itemize}
\item odsetek koordynacji (L-L) względem wszystkich koordynacji (L) \textbf{znacznie rośnie};
\item odsetek koordynacji (0-L) względem wszystkich koordynacji (0) \textbf{rośnie};
\item odsetek koordynacji (R-L) względem wszystkich koordynacji (R) \textbf{nie zmienia się}.
\end{itemize}

Przewidywania modelu \eqref{ED-model} pokrywają się z tendencjami zaobserwowanymi przez \cite{przepiorkowski2023conjunct} oraz przeze mnie w~przypadku języka angielskiego i~języków słowiańskich.

\section{Języki inicjalne}

W językach inicjalnych model przewiduje następujące długości relacji:

\begin{table}[H]
\begin{tabular}{lcllcl}

(L-L) &

\begin{dependency}[hide label, edge unit distance=0.5ex, baseline=-\the\dimexpr\fontdimen22\textfont2\relax]
        \begin{deptext}
        $\odot$\&$\square$\&a\&$\boxdot$\&$\square$\&a+b\\
        \end{deptext}
		\depedge{1}{2}{}
		\depedge{1}{5}{}
		\depedge{5}{4}{}
		\depedge{2}{5}{}
        \wordgroup{1}{2}{3}{L}
        \wordgroup{1}{5}{6}{R}
        \end{dependency}

& $S=2a$ & 

(L-R) &

\begin{dependency}[hide label, edge unit distance=0.5ex, baseline=-\the\dimexpr\fontdimen22\textfont2\relax]
        \begin{deptext}
        $\odot$\&$\square$\&a+b\&$\boxdot$\&$\square$\&a\\
        \end{deptext}
		\depedge{1}{2}{}
		\depedge{1}{5}{}
		\depedge{5}{4}{}
		\depedge{2}{5}{}
		\wordgroup{1}{2}{3}{L}
		\wordgroup{1}{5}{6}{R}
        \end{dependency}
        
& $S=2a+2b$ \\ 

(0-L) &

\begin{dependency}[hide label, edge unit distance=0.5ex, baseline=-\the\dimexpr\fontdimen22\textfont2\relax]
        \begin{deptext}
        $\square$\&a\&$\boxdot$\&$\square$\&a+b\\
        \end{deptext}
		\depedge{4}{3}{}
		\depedge{1}{4}{}
        \wordgroup{1}{1}{2}{L}
        \wordgroup{1}{4}{5}{R}
        \end{dependency}

& $S=a$ & 

(0-R) &

\begin{dependency}[hide label, edge unit distance=0.5ex, baseline=-\the\dimexpr\fontdimen22\textfont2\relax]
        \begin{deptext}
        $\square$\&a+b\&$\boxdot$\&$\square$\&a\\
        \end{deptext}
		\depedge{4}{3}{}
		\depedge{1}{4}{}
        \wordgroup{1}{1}{2}{L}
        \wordgroup{1}{4}{5}{R}
        \end{dependency}
        
& $S=a+b$ \\

(R-L) &

\begin{dependency}[hide label,edge unit distance=0.5ex, baseline=-\the\dimexpr\fontdimen22\textfont2\relax]
        \begin{deptext}
        $\square$\&a\&$\boxdot$\&$\square$\&a+b\&$\odot$\\
        \end{deptext}
		\depedge{6}{1}{}
		\depedge{6}{4}{}
		\depedge{4}{3}{}
		\depedge{4}{1}{}
		\wordgroup{1}{1}{2}{L}
		\wordgroup{1}{4}{5}{R}
        \end{dependency}
        
& $S=4a+2b$ &

(R-R) &

\begin{dependency}[hide label, edge unit distance=0.5ex, baseline=-\the\dimexpr\fontdimen22\textfont2\relax]
        \begin{deptext}
        $\square$\&a+b\&$\boxdot$\&$\square$\&a\&$\odot$\\
        \end{deptext}
		\depedge{6}{1}{}
		\depedge{6}{4}{}
		\depedge{4}{3}{}
		\depedge{4}{1}{}
        \wordgroup{1}{1}{2}{L}
        \wordgroup{1}{4}{5}{R}
        \end{dependency}

& $S=4a+2b$ \\

\end{tabular}
\end{table}

Wraz~ze wzrostem różnicy długości członów (\emph{b}):
\begin{itemize}
\item odsetek koordynacji (L-L) względem wszystkich koordynacji (L) \textbf{znacznie rośnie};
\item odsetek koordynacji (0-L) względem wszystkich koordynacji (0) \textbf{rośnie};
\item odsetek koordynacji (R-L) względem wszystkich koordynacji (R) \textbf{nie zmienia się}.
\end{itemize}

Przewidywania modelu \eqref{ED-model} pokrywają się z tendencjami zaobserwowanymi przez \cite{przepiorkowski2023conjunct} oraz przeze mnie w~przypadku języka angielskiego i~języków słowiańskich.


