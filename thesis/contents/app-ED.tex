\chapter{Rozszerzone zależności składniowe}

\section{Enhanced Dependencies}

W niniejszej pracy analizuję relacje składniowe w~formie drzew zależnościowych. Oznacza to, że zakładam, że relacje składniowe w~zdaniach tworzą drzewa. Jest to dość powszechne w~językoznawstwie założenie, stojące u podstaw gramatyki zależnościowej. Jednak nie musi być ono prawdziwe. 

Jak zauważają \cite{de2021universal}, w~języku istnieje wiele konstrukcji, w~przypadku których opis relacji zależnościowych za pomocą drzew nie jest optymalny. Należą do nich m.in. koordynacje.

Z~tego powodu twórcy Universal Dependencies stworzyli rozszerzenie standardu o~nazwie Enhanced Dependencies (ED)\footnote{\url{https://universaldependencies.org/u/overview/enhanced-syntax.html}}. Przykład \eqref{UD} pokazuje drzewo zależnościowe zdania \eqref{tworzyć+konstruować} opisane według zwykłego UD. \eqref{ED} jest drzewem zdania \eqref{tworzyć+konstruować} opisanym według ED. Rozszerzone zależności zaznaczone są na niebiesko.

\begin{exe}
\ex \label{tworzyć+konstruować}
Chcą tworzyć, konstruować.
 
\ex \label{UD}
\begin{dependency}[baseline=-\the\dimexpr\fontdimen22\textfont2\relax]
\begin{deptext}[column sep=1em]
Chcą \& tworzyć \& , \& konstruować \& .  \\ 
\end{deptext}
\deproot{1}{root}
\depedge{1}{2}{xcomp}
\depedge{4}{3}{punct}
\depedge{2}{4}{conj}
\depedge{1}{5}{punct}
\end{dependency}

\ex \label{ED}
\begin{dependency}[baseline=-\the\dimexpr\fontdimen22\textfont2\relax]
\begin{deptext}[column sep=1em]
Chcą \& tworzyć \& , \& konstruować \& .  \\ 
\end{deptext}
\deproot{1}{root}
\depedge{1}{2}{xcomp}
\depedge{4}{3}{punct}
\depedge{2}{4}{conj}
\depedge{1}{5}{punct}
\depedge[edge style = {blue, thick}, label style = {thick, draw=blue, text=blue}]{1}{4}{xcomp}
\end{dependency}
\citep{przepiorkowski2018lexical}
\end{exe}

\cite{przepiorkowski2023conjunct} zauważają, że uwzględnienie rozszerzonych relacji zależnościowych w~strukturze koordynacji zmienia przewidywania modeli dotyczące omawianych tendencji. Prezentują model struktury zależnościowej koordynacji oparty na założeniach ED:

\begin{exe}
\ex \label{ED-model}
\begin{dependency}[theme = simple, edge unit distance=0.5ex, baseline=-\the\dimexpr\fontdimen22\textfont2\relax]
        \begin{deptext}
        $\odot$ \& $\square$ \& $\boxdot$ \& $\square$\\
            \end{deptext}
		\depedge{1}{2}{}
		\depedge{1}{4}{}
		\depedge{2}{4}{}
		\depedge{4}{3}{}
        \end{dependency}
\end{exe}

\subsection{Języki finalne}

W językach finalnych model przewiduje następujące długości relacji:

\begin{table}[H]
\begin{tabular}{lcllcl}

(L-L) &

\begin{dependency}[hide label, edge unit distance=0.5ex, baseline=-\the\dimexpr\fontdimen22\textfont2\relax]
        \begin{deptext}
        $\odot$\&a\&$\square$\&$\boxdot$\&a+b\&$\square$\\
        \end{deptext}
		\depedge{1}{3}{}
		\depedge{1}{6}{}
		\depedge{6}{4}{}
		\depedge{3}{6}{}
        \wordgroup{1}{2}{3}{L}
        \wordgroup{1}{5}{6}{R}
        \end{dependency}

& $S=5a+3b$ & 

(L-R) &

\begin{dependency}[hide label, edge unit distance=0.5ex, baseline=-\the\dimexpr\fontdimen22\textfont2\relax]
        \begin{deptext}
        $\odot$\&a+b\&$\square$\&$\boxdot$\&a\&$\square$\\
        \end{deptext}
		\depedge{1}{3}{}
		\depedge{1}{6}{}
		\depedge{6}{4}{}
		\depedge{3}{6}{}
		\wordgroup{1}{2}{3}{L}
		\wordgroup{1}{5}{6}{R}
        \end{dependency}
        
& $S=5a+2b$ \\ 

(0-L) &

\begin{dependency}[hide label, edge unit distance=0.5ex, baseline=-\the\dimexpr\fontdimen22\textfont2\relax]
        \begin{deptext}
        a\&$\square$\&$\boxdot$\&a+b\&$\square$\\
        \end{deptext}
		\depedge{5}{3}{}
		\depedge{2}{5}{}
        \wordgroup{1}{1}{2}{L}
        \wordgroup{1}{4}{5}{R}
        \end{dependency}

& $S=2a+2b$ & 

(0-R) &

\begin{dependency}[hide label, edge unit distance=0.5ex, baseline=-\the\dimexpr\fontdimen22\textfont2\relax]
        \begin{deptext}
        a+b\&$\square$\&$\boxdot$\&a\&$\square$\\
        \end{deptext}
		\depedge{5}{3}{}
		\depedge{2}{5}{}
        \wordgroup{1}{1}{2}{L}
        \wordgroup{1}{4}{5}{R}
        \end{dependency}
        
& $S=2a$ \\

(R-L) &

\begin{dependency}[hide label,edge unit distance=0.5ex, baseline=-\the\dimexpr\fontdimen22\textfont2\relax]
        \begin{deptext}
        a\&$\square$\&$\boxdot$\&a+b\&$\square$\&$\odot$\\
        \end{deptext}
		\depedge{6}{2}{}
		\depedge{6}{5}{}
		\depedge{5}{3}{}
		\depedge{5}{2}{}
		\wordgroup{1}{1}{2}{L}
		\wordgroup{1}{4}{5}{R}
        \end{dependency}
        
& $S=3a+3b$ &

(R-R) &

\begin{dependency}[hide label, edge unit distance=0.5ex, baseline=-\the\dimexpr\fontdimen22\textfont2\relax]
        \begin{deptext}
           a+b\&$\square$\&$\boxdot$\&a\&$\square$\&$\odot$\\
        \end{deptext}
		\depedge{6}{2}{}
		\depedge{6}{5}{}
		\depedge{5}{3}{}
		\depedge{5}{2}{}
        \wordgroup{1}{1}{2}{L}
        \wordgroup{1}{4}{5}{R}
        \end{dependency}

& $S=3a$ \\

\end{tabular}
\end{table}

Wraz~ze wzrostem różnicy długości członów (\emph{b}):
\begin{itemize}
\item odsetek koordynacji (L-L) względem wszystkich koordynacji (L) \textbf{spada};
\item odsetek koordynacji (0-L) względem wszystkich koordynacji (0) \textbf{znacznie spada};
\item odsetek koordynacji (R-L) względem wszystkich koordynacji (R) \textbf{znacznie spada}.
\end{itemize}

Przewidywania modelu \eqref{ED-model} nie pokrywają się z uzyskanymi wynikami dla języka koreańskiego i~tureckiego. Oznacza to, że model oparty na~rozszerzonych zależnościach składniowych nie wyjaśnia uzyskanych wyników\footnote{
W przypadku języków finalnych należy rozpatrzeć też podejście zakładające, że spójnik jest podrzędnikiem prawego członu. W przypadku tego wariantu predykcje również nie pokrywają się z uzyskanymi wynikami.}.

\subsection{Języki finalne}

W językach finalnych model przewiduje następujące długości relacji:

\begin{table}[H]
\begin{tabular}{lcllcl}

(L-L) &

\begin{dependency}[hide label, edge unit distance=0.5ex, baseline=-\the\dimexpr\fontdimen22\textfont2\relax]
        \begin{deptext}
        $\odot$\&a\&$\square$\&$\boxdot$\&a+b\&$\square$\\
        \end{deptext}
		\depedge{1}{3}{}
		\depedge{1}{6}{}
		\depedge{6}{4}{}
		\depedge{3}{6}{}
        \wordgroup{1}{2}{3}{L}
        \wordgroup{1}{5}{6}{R}
        \end{dependency}

& $S=5a+3b$ & 

(L-R) &

\begin{dependency}[hide label, edge unit distance=0.5ex, baseline=-\the\dimexpr\fontdimen22\textfont2\relax]
        \begin{deptext}
        $\odot$\&a+b\&$\square$\&$\boxdot$\&a\&$\square$\\
        \end{deptext}
		\depedge{1}{3}{}
		\depedge{1}{6}{}
		\depedge{6}{4}{}
		\depedge{3}{6}{}
		\wordgroup{1}{2}{3}{L}
		\wordgroup{1}{5}{6}{R}
        \end{dependency}
        
& $S=5a+2b$ \\ 

(0-L) &

\begin{dependency}[hide label, edge unit distance=0.5ex, baseline=-\the\dimexpr\fontdimen22\textfont2\relax]
        \begin{deptext}
        a\&$\square$\&$\boxdot$\&a+b\&$\square$\\
        \end{deptext}
		\depedge{5}{3}{}
		\depedge{2}{5}{}
        \wordgroup{1}{1}{2}{L}
        \wordgroup{1}{4}{5}{R}
        \end{dependency}

& $S=2a+2b$ & 

(0-R) &

\begin{dependency}[hide label, edge unit distance=0.5ex, baseline=-\the\dimexpr\fontdimen22\textfont2\relax]
        \begin{deptext}
        a+b\&$\square$\&$\boxdot$\&a\&$\square$\\
        \end{deptext}
		\depedge{5}{3}{}
		\depedge{2}{5}{}
        \wordgroup{1}{1}{2}{L}
        \wordgroup{1}{4}{5}{R}
        \end{dependency}
        
& $S=2a$ \\

(R-L) &

\begin{dependency}[hide label,edge unit distance=0.5ex, baseline=-\the\dimexpr\fontdimen22\textfont2\relax]
        \begin{deptext}
        a\&$\square$\&$\boxdot$\&a+b\&$\square$\&$\odot$\\
        \end{deptext}
		\depedge{6}{2}{}
		\depedge{6}{5}{}
		\depedge{5}{3}{}
		\depedge{5}{2}{}
		\wordgroup{1}{1}{2}{L}
		\wordgroup{1}{4}{5}{R}
        \end{dependency}
        
& $S=3a+3b$ &

(R-R) &

\begin{dependency}[hide label, edge unit distance=0.5ex, baseline=-\the\dimexpr\fontdimen22\textfont2\relax]
        \begin{deptext}
           a+b\&$\square$\&$\boxdot$\&a\&$\square$\&$\odot$\\
        \end{deptext}
		\depedge{6}{2}{}
		\depedge{6}{5}{}
		\depedge{5}{3}{}
		\depedge{5}{2}{}
        \wordgroup{1}{1}{2}{L}
        \wordgroup{1}{4}{5}{R}
        \end{dependency}

& $S=3a$ \\

\end{tabular}
\end{table}

Wraz~ze wzrostem różnicy długości członów (\emph{b}):
\begin{itemize}
\item odsetek koordynacji (L-L) względem wszystkich koordynacji (L) \textbf{spada};
\item odsetek koordynacji (0-L) względem wszystkich koordynacji (0) \textbf{znacznie spada};
\item odsetek koordynacji (R-L) względem wszystkich koordynacji (R) \textbf{znacznie spada}.
\end{itemize}

Przewidywania modelu \eqref{ED-model} nie pokrywają się z uzyskanymi wynikami dla języka koreańskiego i~tureckiego. Oznacza to, że model oparty na~rozszerzonych zależnościach składniowych nie wyjaśnia uzyskanych wyników\footnote{
W przypadku języków finalnych należy rozpatrzeć też podejście zakładające, że spójnik jest podrzędnikiem prawego członu. W przypadku tego wariantu predykcje również nie pokrywają się z uzyskanymi wynikami.}.


